%instiki:category: FisicaSubatomica
%http://localhost:2500/wiki/show/Cap%C3%ADtulo+II
\chapter{Campos bosónicos}
\label{cha:campos-vectoriales} %noinstiki
%instiki:
%instiki:***
%instiki:
%instiki:[[NotasFS|Tabla de Contenidos]]
%instiki:
%instiki:***
%generated with html2itexTOC instiki_source.html
%instiki:
%instiki:* [Unidades Naturales](#NU)
%instiki:
%instiki:* [Notaci\'on relativista](#srn)
%instiki:
%instiki:  * [Ejemplos de cuadrivectores](#ejemplos_de_cuadrivectores)
%instiki:
%instiki:  * [Ecuaciones covariantes](#ecuac-covar)
%instiki:
%instiki:* [Ecuaciones de Maxwell en notaci\'on covariante](#maxeqs)
%instiki:
%instiki:  * [Lagrangiano Electromagn\'etico](#lagr-electr)
%instiki:
%instiki:  * [Energ\'\i a del campo electromagn\'etico](#energa_del_campo_electromagntico)
%instiki:
%instiki:  * [Fijaci\'on del gauge](#fijacion-del-gauge)
%instiki:
%instiki:* [Ecuaciones de Proca](#ecuacion-de-proca)
%instiki:
%instiki:* [Problemas](#problemas2)
%instiki:
%instiki:***
%instiki:

Se hará un repaso de las nociones de relatividad especial para mostrar como se transforman los campos escalares y vectoriales bajo transformaciones de Lorentz. Se mostrará como la invarianza de la Acción bajo este tipo de transformación es el punto de partida en la construcción de densidades Lagrangianas únicas.


\section{Unidades Naturales}
\label{sec:NU}
Las \emph{unidades naturales} son unidades f\'\i sicas de medida definidas en t\'erminos de constantes f\'\i sicas universales~\cite{NU}. El primer conjunto consiste de unidades naturales, las \emph{unidades de Planck}~\cite{PU},  fue formulado por el propio Planck despu\'es de establecer la \'ultima constante universal, que lleva su nombre.  En palabras de Planck
\begin{quotation} %noinstiki

  ...ihre Bedeutung f\"ur alle Zeiten und f\"ur alle, auch au\ss erirdische und au\ss ermenschliche Kulturen notwendig behalten und welche daher als \guillemotright nat\"urliche Ma\ss einheiten bezeichnet werden k\"onnen... %noinstiki

...These necessarily retain their meaning for all times and for all civilizations, even extraterrestrial and non-human ones, and can therefore be designated as 
``natural units''... %noinstiki
%instiki: natural units
\end{quotation} %noinstiki
%instiki:
De este modo, estas unidades son naturales debido a que el origen de su definici\'on proviene solo de propiedades de la naturaleza y no de alguna construcci\'on humana. A diferencia de otros conjuntos de unidades naturales las unidades de Planck donde
\begin{equation}
\label{eq:144}
  G_N=1,\qquad \hbar=1,\qquad c=1,\qquad K=\frac{1}{4\pi\epsilon_0}=1,\qquad k=1,
\end{equation}
est\'an basadas s\'olo en las propiedades del espacio libre, y no en las propiedades (tales como carga, masa, tama\~no o radio) de alg\'un objeto o part\'\i cula elemental.

Las constantes f\'\i sicas que suelen normalizarse se escogen del conjunto dado por la ec.~\eqref{eq:144} y
\begin{equation}
  \label{eq:145}
  e,\qquad m_e,\qquad m_p.
\end{equation}
Teniendo en cuenta que $1\,\text{eV}=1.602\;176\;487(40)\times10^{-19}\,\text{J}$.
\begin{align}
  10^{-9}\,\text{GeV}=&1.602\;176\;487(40)\times10^{-19}\,\text{J}\nonumber\\
  1\,\text{GeV}=&1.602\;176\;487(40)\times10^{-10}\,\text{J}\nonumber\\
\end{align}

\begin{example}
  Calcule la energ\'\i a cinetica de un mosquito de $2\,$mg, moviendos a $1.6\,$Km/h
\begin{verbatim}
V = 1.6 x 10-19 Joules

1 TeV = 1.6 x 10-19 x 1012 Joules = 1.6 x 10-7 Joules

1/2 m v2 = 1.6 x 10-7 Joules,  m = 2 x 10-6 kg therefore v = 0.4 m/s  = 1.4 kph
\end{verbatim}


\end{example}

Teniendo en cuenta que \cite{PDG}
\begin{equation}
    c=299\;792\;450\,\text{m}\,\text{s}^{-1}\qquad\text{(exact)}\,,
\end{equation}
podemos obtener la relaci\'on entre longitud y energ\'\i a a partir de
\begin{align}
  \hbar c=&1.054\;571\;68(53)\times10^{-34}\,\text{J}\,\text{s}\times299\;792\;450\,\text{m}\,\text{s}^{-1} \nonumber\\
  \approx&3.161\;526\;28\times10^{-26}\,\text{J}\,\text{m}\nonumber\\
  \approx&3.161\;526\;28\times10^{-26}\,{\text{J}}\frac{1\,\text{GeV}}{1.602\;176\;487\times10^{-10}\,\text{J}}\,\text{m}\nonumber\\
  =&1.973\;269\;631(49)\times10^{-16}\,\text{GeV}\,\text{m}.
\end{align}
Entonces $\hbar c =0.1973\;269\;631(49)\,\text{GeV}\,\text{fm}$.
\begin{example}
  Calcule la energ\'\i a potencial de Coulomb para una par de protones (o electrones) separados una distancia $l=\hbar\,c/\text{GeV}$ $=0.1973\;269\;631\,\text{fm}$
  \begin{align}
    \label{eq:240}
    V=\frac{k e^2}{l}=&\frac{e^2}{4\pi\epsilon_0(\hbar c)\,\text{GeV}^{-1}}\nonumber\\
    =&\frac{e^2}{4\pi\epsilon_0\hbar c}\text{GeV}\,.
  \end{align}
\end{example}
Como $V$ tiene unidades de energ\'\i a, de la ec.~(\ref{eq:240})  resulta entonces la constante adimensional 
conocida como la constante de estructura fina
\begin{equation}
  \alpha=\frac{e^2}{4\pi\epsilon_0\hbar c},
\end{equation}
que no puede tomar un valor num\'erico diferente  sin importar el sistema de unidades que se use. De modo que no se puede tener un sistema de unidades que normalice todas las contanstes f\'\i sicas presentes en $\alpha$. S\'olo 3 de las cuatro constantes $e$, $\hbar$, $\epsilon_0$ y $c$ pueden ser normalizadas, y la otra queda dependiendo del valor de $\alpha$.

El prop\'osito de las unidades naturales es simplificar las expresiones algebraicas que aparecen en las leyes f\'\i sicas. 

El sistema de unidades naturales que usaremos es el de las Unidades de Planck Modificadas (MPU)
\begin{equation}
  G_N=1,\qquad \hbar=1 \qquad c=1,\qquad \epsilon_0=1,\qquad k=1,
\end{equation}
de modo que
\begin{equation}
  e=\sqrt{4\pi\alpha},\qquad\text{or}\qquad \alpha=\frac{e^2}{4\pi}.
\end{equation}
podemos obtener la relaci\'on entre el tiempo y la energ\'\i a de
\begin{equation}
  \hbar\equiv\frac{h}{2\pi}=1.054\;571\;68(53)\times10^{-34}\,\text{J}\,\text{s}
  =6.582\;118\;99(16)\times10^{-25}\,\text{GeV}\,\text{s},
\end{equation}
Similarmente para la relaci\'on entre temperatura y energ\'\i a, tenemos de la constante de Boltzman
\begin{equation}
  k=1.380\;6504(24)\times10^{-23}\,\text{J}\,\text{K}^{-1}=8.617\;343(15)\times10^{-14}\,\text{GeV}\,\text{K}^{-1}.
\end{equation}

La relaci\'on ente masa y energ\'\i a se puede obtener a partir de 
\begin{equation}
  G_N=6.674\;28(67)\times10^{-11}\text{m}^3\text{kg}^{-1}\,\text{s}^{-2}=6.70881(65)\times10^{-39}\hbar c(\text{GeV}/c^2)^{-2}
\end{equation}
entonces\footnote{o de una masa bien medida, por ejemplo
  $m_p=0.938\;272\;013(23)\,\text{GeV}/c^2=1.672\;621\;637(83)^{-27}\,\text{kg}.$
}
\begin{equation}
  M_p\equiv\sqrt{\frac{\hbar c}{G_N}}=1.2209\times10^{19}\text{GeV}/c^2=2.1765\times10^{-8}\,\text{kg}\,
\end{equation}
y
\begin{align}
  \label{eq:242}
  G_N=\frac{\hbar\,c}{M_p^2}\,.
\end{align}

La energ\'\i a de Planck es entonces $M_p c^2$, e igual a la masa en unidad naturales.
Los factores de conversi\'on del sistema MKS a MPU est\'an dados en la Tabla~\ref{tab:mks2mpu} despu\'es de hacer $\hbar=c=k=1$

\begin{table} %noinstiki
  \centering %noinstiki
  \begin{tabular}{c|c} %noinstiki
%instiki:
$6.582\;118\;99(16)\times10^{-25}\,\text{s}$ & $ {\hbar}\,\text{GeV}^{-1}$\\\hline
%instiki:
$1.973\;269\;631(49)\times10^{-16}\,\text{m}$ & $ {\hbar c}\,\text{GeV}^{-1} $\\ \hline
%instiki:
1\,kg& $5.609\;589\;12(42)\times10^{26}\,\text{GeV}/c^2$ \\ \hline
%instiki:
1\,K & $8.617\;343(15)\times10^{-14}/k\,\text{GeV}$\,\\ \hline
$299\;792\;450\,\text{m}\,\text{s}^{-1}$&$c$\\ \hline
%instiki:
m\,kg&$2.842\;278\,859\times10^{-16}\hbar\,c^{-1}$\\ \hline
%instiki:
  \end{tabular} %noinstiki
  \caption{MKS $\leftrightarrow$ MPU} %noinstiki
  \label{tab:mks2mpu} %noinstiki
\end{table} %noinstiki

De los factores de conversi\'on de la tabla vemos que masa$\times$longitud tiene las mismas unidades que $\hbar/c$, de modo que podemos definir la longitud de Planck tal que
\begin{align}
  L_p\,M_p\equiv&\frac{\hbar}{c}\nonumber\\
  L_p=&\frac{\hbar}{c\,M_p}=\frac{\hbar}{c}\sqrt{\frac{G_N}{\hbar c}}=\sqrt{\frac{\hbar\, G_N}{c^3}}\nonumber\\
  &\approx8.1907\times10^{-20}\frac{\hbar}{c\,\text{GeV}} \approx1.6163\times10^{-35}\,\text{m}
\end{align}

Este an\'alisis dimensional muestra que la longitud de Planck corresponde a una escala a la cual los efectos gravitacionales llegan a ser importantes, es decir, que la intensidad del potencial gravitacional es del orden de la masa de la part\'\i cula que lo genera\footnote{This is shown using dimensional analysis, much in the same
way as the Bohr radius, beyond which the full quantum mechanical
description of the Hydrogen atom cannot be neglected. Note that the
Bohr radius was derived before a modern quantum mechanical treatment
of Hydrogen became available. A similar statement can be made about
the Planck length \cite{andim}.}
\begin{equation}
  \label{eq:241}
  V_{\text{gravity}}=G_N\frac{M_p^2}{L_p}=M_pc^2
\end{equation}


Finalmente, el tiempo de Planck es
\begin{equation}
  t_P\equiv\frac{L_p}{c}=\sqrt{\frac{\hbar\, G_N}{c^5}}\approx8.1907\times10^{-20}\frac{\hbar}{c^2\,\text{GeV}}\approx5.3912\times10^{-44}\,\text{s}\,,
\end{equation}
y la temperatura de Planck es
\begin{equation}
  T_p\equiv\frac{M_p c^2}{k}=\sqrt{\frac{\hbar c^5}{G_N\,k^2}}=1.4168\times10^{32}\,\text{K}\,.
\end{equation}
Teniendo en cuenta la condici\'on en (\ref{eq:241}), podemos tambi\'en definir la carga de Planck tal que la intensidad de Potencial de Coulomb para dos masas de Planck separadas por la longitud de Planck sea igual a la energ\'\i a de Planck
\begin{align}
  V_{\text{Coulomb}}=\frac{1}{4\pi\epsilon_0}\frac{Q_p^2}{L_p}=&M_pc^2\nonumber\\
  \frac{1}{4\pi\epsilon_0}\frac{Q_p^2}{\hbar/(c\,M_p)}=&M_p\,c^2\nonumber\\
  \frac{Q_p^2}{4\pi\epsilon_0\hbar c}=&1\nonumber\\
  Q_p=&\frac{e}{\sqrt{\alpha}}\approx1.8756\times10^{-18}\,\text{C}.
\end{align}
Entonces la constante de estructura fina puede pensarse como el cuadrado del cociente de la carga elemental a la carga de Planck
\begin{align}
  \alpha=\left(\frac{e}{Q_p}\right)^2\,.
\end{align}
Estos resultados est\'an resumidos en la Tabla~\ref{tab:PU}
\begin{table} %noinstiki
  \centering %noinstiki
  \begin{tabular}{c|c|c} %noinstiki
%instiki:
$M_p$&$\sqrt{\hbar c/G_N}$&$2.1765\times10^{-8}\,\text{kg}$\\
%instiki:
$L_p$&$\sqrt{{\hbar\, G_N}/{c^3}}$&$1.6163\times10^{-35}\,\text{m}$\\
%instiki:
$t_P$&$\sqrt{{\hbar\, G_N}/{c^5}}$&$5.3912\times10^{-44}\,\text{s}$\\
%instiki:
$T_p$&$\sqrt{{\hbar c^5}/(G_N\,k^2)}$&$1.4168\times10^{32}\,\text{K}$\\
%instiki:
$Q_p$&${e}/{\sqrt{\alpha}}$&$1.8756\times10^{-18}\,\text{C}$\\
%instiki:
  \end{tabular} %noinstiki
  \caption{Unidades de Planck $G_N=\hbar=c=\epsilon_0=k=1$} %noinstiki
  \label{tab:PU} %noinstiki
\end{table} %noinstiki

\begin{example}
  C\'alcule el potencial gravitacional para un par de protones separados una distancia  $L_{\text{proton}}=\hbar/(c\, m_{\text{proton}})\approx2.1\times10^{-16}\,\text{m}$
\begin{equation}
    V_{\text{gravity}}=G_N\frac{m_p^2}{L_{\text{proton}}}=G_N m_p^3\frac{c}{\hbar}=\frac{G_N}{\hbar c}m_p^3c^2=\frac{m_p^2}{M_p^2}m_pc^2\approx10^{-38}m_pc^2
\end{equation}
En este caso la energ\'\i a potencial gravitacional es mucho menor que la escala de energ\'\i a correspondiente
\end{example}
\begin{example}
Compare la intensidad gravitacional con la Coulomb para un prot\'on, y para una part\'\i cula de Planck.

Usando la ec.~(\ref{eq:242})
\begin{align}
\frac{V_{\text{gravity}}}{V_{\text{Coulomb}}}=&\frac{G_N\,m_{\text{X}}^2}{(1/4\pi\epsilon_0)e^2}\nonumber\\
=&(4\pi\epsilon_0\hbar\,c/e^2)\frac{m_{\text{X}}^2}{M_p^2}\nonumber\\
=&\frac{1}{\alpha}\left(\frac{m_{\text{X}}}{M_p}\right)^2\nonumber\\
\sim&
\begin{cases}
  10^{-36}&m_{\text{X}}=m_{\text{proton}}\\
  10^{2}&m_{\text{X}}=M_{\text{p}}\\
\end{cases}
\end{align}
\end{example}


\begin{example}
De la constante de Fuerza electrost\'atica $K=1/(4\pi\epsilon_0)$, podemos obtener el valor de la constante de estructura fina electromagn\'etica $\alpha=e^2/(4\pi\epsilon_0\hbar c)$

\begin{align*}
  K=\frac{1}{4\pi\epsilon_0}\approx&\frac{1}{4\pi\times8.854\times10^{-12}}\text{C}^{-2}\text{Nm}^2
  =\frac{1}{4\pi\times8.854\times10^{-12}}\text{C}^{-2}\text{Kg}\,\text{m}^3\text{s}^{-2}\\
  \approx&\frac{1}{4\pi\times8.854\times10^{-12}}\text{C}^{-2}\times5.6096\times10^{26}\text{GeV}
  \times(5.068\times10^{15}\text{GeV}^{-1})^3\\
  &\times(1.519\times10^{-24}\text{GeV}^{-1})^{-2}\times\frac{(\hbar c)^3\hbar^{-2}}{c^2}\\
  \approx&2.84\times10^{35}\text{C}^{-2}\hbar c\\
  \approx&2.84\times10^{35}\text{C}^{-2}\times
  \left(
    \frac{1.602\times10^{-19}}{e^2}
  \right)^2\hbar c\\
  =&\frac{7.296\times10^{-3}}{e^2}\hbar c
\end{align*}
Definimos la cantidad adimensional $\alpha$, como
\begin{equation*}
  \alpha\equiv\frac{e^2}{4\pi\epsilon_0\hbar c}
\approx7.296\times10^{-3}\approx\frac{1}{137}
\end{equation*}
  
\end{example}


\section{Notaci\'on relativista}
\label{sec:srn}
Las transformaciones de Lorentz se definen como la transformaciones que dejan invariante al producto escalar en el espacio de Minkowski definido como
\begin{equation}
  \label{eq:146}
  a^2=g_{\mu\nu}a^\mu a^\nu\equiv a_\nu a^\nu={a^0}^2-a^i a^i={a^0}^2-\mathbf{a}\cdot\mathbf{a}
\end{equation}
donde $\mu,\nu=0,1,2,3$, $i=1,2,3$ y se asume suma sobre \'\i ndices repetidos. Adem\'as
\begin{equation}
\label{eq:149}
  a_\nu\equiv g_{\mu\nu}a^\mu
\end{equation}
 Finalmente la m\'etrica usada se define como
\begin{equation}
  \label{eq:gmunu}
  \left\{ g_{\mu\nu} \right\}=
  \begin{pmatrix}
    1&0&0&0\\
    0&-1&0&0\\
    0&0&-1&0\\
    0&0&0&-1
  \end{pmatrix}
\end{equation}
donde $\left\{ g_{\mu\nu} \right\}$ denota la forma matricial del tensor $g_{\mu\nu}$.  

El producto de dos cuadrivectores se define en forma similar como
\begin{equation}
\label{eq:157}
  a_\nu b^\nu=g_{\mu\nu}a^\mu b^\nu=a^0b^0-\mathbf{a}\cdot\mathbf{b}
\end{equation}
El inverso de la m\'etrica es
\begin{equation}
  \left\{ g^{\mu\nu} \right\}\equiv\left\{ g_{\mu\nu} \right\}^{-1}=\left\{ g_{\mu\nu} \right\}
\end{equation}
tal que
\begin{equation}
  g^{\mu\alpha}g_{\alpha\nu}=\delta^\mu_\nu\qquad\text{and}\qquad a^\mu=g^{\mu\nu}a_\nu
\end{equation}

Bajo una transformaci\'on de Lorentz.
\begin{equation}
  a^\mu\to {a'}^\mu={\Lambda^\mu}_{\nu}a^\nu.
\end{equation}
La invarianza del producto escalar en ec.~\eqref{eq:157}
\begin{equation}
  {a'}^\mu{b'}_\mu=a^\mu b_\mu
\end{equation}
da lugar a
\begin{equation}
  \label{eq:lrinv}
  g_{\mu\nu}={\Lambda^\alpha}_{\mu}g_{\alpha\beta}{\Lambda^\beta}_{\nu}\qquad\text{or}\qquad 
\left\{g_{\mu\nu}\right\}=\left\{{\Lambda_{\mu}}^{\alpha}\right\}^{\text{T}}\left\{g_{\alpha\beta}\right\}\left\{{\Lambda^\beta}_{\nu}\right\}.
\end{equation}
En notaci\'on matricial
\begin{align}
 g=\Lambda^T g \Lambda\,. 
\end{align}

\begin{english}
From eq.~\eqref{eq:lrinv} we also have  
\end{english}
\begin{spanish}
De la ec.~\eqref{eq:lrinv} tenemos que
\end{spanish}

\begin{align}
  g^{\rho\mu}g_{\mu\nu}=&g^{\rho\mu}{\Lambda^\alpha}_{\mu}g_{\alpha\beta}{\Lambda^\beta}_{\nu}\nonumber\\
  \delta^\rho_\nu=&{\Lambda_\beta}^\rho{\Lambda^\beta}_{\nu}\,,
\end{align}
\begin{english}
or  
\end{english}
\begin{spanish}
o
\end{spanish}
\begin{align}
  {\Lambda_\alpha}^\mu{\Lambda^\alpha}_{\nu}=\delta^\mu_\nu\,.
\end{align}
\begin{english}
Since
\end{english}
\begin{spanish}
Ya que
\end{spanish}
\begin{align}
  {\left(\Lambda^{-1}\right)^\mu}_\alpha{\Lambda^\alpha}_{\nu}=\delta^\mu_\nu\,
\end{align}
\begin{english}
the inverse of $\Lambda$ is
\end{english}
\begin{spanish}
el inverso de $\Lambda$ es
\end{spanish}
\begin{align}
  {\left(\Lambda^{-1}\right)^\mu}_\alpha={\Lambda_\alpha}^\mu\,,
\end{align}
\begin{english}
or  
\end{english}
\begin{spanish}
o
\end{spanish}
\begin{align}
\label{eq:lambdainv}
  {\left(\Lambda^{-1}\right)^\mu}_\nu={\Lambda_\nu}^\mu\,,
\end{align}
\begin{itemize}
\item
\begin{english}
\textbf{Example:} Lorentz invariance    
\end{english}
\begin{spanish}
\textbf{Ejemplo:} Invarianza de Lorentz
\end{spanish}
  \begin{align}
    a_\mu b^\mu\to a'_\mu{b'}^\mu=&{\Lambda_\mu}^\nu a_\nu{\Lambda^\mu}_\rho b^p \nonumber\\
    =&{\Lambda_\mu}^\nu a_\nu{\Lambda^\mu}_\rho b^p \nonumber\\
    =&{\left(\Lambda^{-1}\right)^\nu}_\mu{\Lambda^\mu}_\rho a_\nu b^p \nonumber\\
    =&\delta^\nu_\rho a_\nu b^p \nonumber\\
    =&a_\nu b^\nu \nonumber\,.
  \end{align}

\end{itemize}

Como un ejemplo de Transformaci\'on de Lorentz considere un desplazamiento a lo largo del eje $x$
\begin{equation}
\label{eq:147}
  \left\{x^\mu\right\}=\begin{pmatrix}
    t\\
    x\\
    y\\
    z
  \end{pmatrix}\to
  \begin{pmatrix}
    t'\\
    x'\\
    y'\\
    z'
  \end{pmatrix}=
  \begin{pmatrix}
    \frac{t+vx}{\sqrt{1-v^2}}\\
    \frac{x+vt}{\sqrt{1-v^2}}\\
    y\\
    z
  \end{pmatrix}=
  \begin{pmatrix}
    \cosh\xi&\sinh\xi&0&0\\
    \sinh\xi&\cosh\xi&0&0\\
    0     &  0  &1&0\\
    0     &  0  &0&1
  \end{pmatrix}
  \begin{pmatrix}
    t\\
    x\\
    y\\
    z
  \end{pmatrix}=\left\{{\Lambda^\mu}_{\nu}\right\}\left\{x^\nu\right\},
\end{equation}
donde
\begin{equation}
  \cosh\xi=\gamma\qquad\sinh\xi=v\gamma,\qquad\text{and}\qquad \gamma=\frac{1}{\sqrt{1-v^2}}.
\end{equation}
y, por ejemplo:
\begin{align}
  t\cosh{\xi}+x\sinh\xi=\gamma(t+v x)=\frac{t+v x}{\sqrt{1-v^2}}\,.
\end{align}
El ${\Lambda^\mu}_{\nu}$ definido en la ec.~\eqref{eq:147} satisface la condici\'on en ec.~\eqref{eq:lrinv}, 
\begin{align}
  \Lambda^T g \Lambda=&\begin{pmatrix}
    \cosh\xi&\sinh\xi&0&0\\
    \sinh\xi&\cosh\xi&0&0\\
    0     &  0  &1&0\\
    0     &  0  &0&1
  \end{pmatrix}
  \begin{pmatrix}
    1 & 0  & 0 &0\\
    0 & -1 & 0 &0\\
    0 & 0  & -1&0\\
    0 & 0  & 0 &-1\\
  \end{pmatrix}
  \begin{pmatrix}
    \cosh\xi&\sinh\xi&0&0\\
    \sinh\xi&\cosh\xi&0&0\\
    0     &  0  &1&0\\
    0     &  0  &0&1
  \end{pmatrix}\nonumber\\
  =&\begin{pmatrix}
       \cosh\xi&-\sinh\xi&0&0\\
    \sinh\xi&-\cosh\xi&0&0\\
    0     &  0  &-1&0\\
    0     &  0  &0&-1
  \end{pmatrix}
 \begin{pmatrix}
    \cosh\xi&\sinh\xi&0&0\\
    \sinh\xi&\cosh\xi&0&0\\
    0     &  0  &1&0\\
    0     &  0  &0&1
  \end{pmatrix}\nonumber\\
  =&\begin{pmatrix}
       \cosh^2\xi-\sinh^2\xi&\cosh\xi\sinh\xi-\cosh\xi\sinh\xi&0&0\\
    \cosh\xi\sinh\xi-\cosh\xi\sinh\xi&\sinh^2\xi-\cosh^2\xi&0&0\\
    0     &  0  &-1&0\\
    0     &  0  &0&-1
  \end{pmatrix}\nonumber\\
=&g
\end{align}

Denotaremos los cuadrivectores con \'\i ndices arriba como
\begin{equation}
  \label{eq:upindx}
  a^\mu=(a^0,a^1,a^2,a^3)=(a^0,\mathbf{a})
\end{equation}
Entonces el correspondiente cuadrivector con \'\i ndices abajo, usando la ec.~\eqref{eq:149}, es
\begin{equation}
  a_\mu=(a_0,a_1,a_2,a_3)=(a^0,-a^1,-a^2,-a^3)=(a^0,-\mathbf{a}).
\end{equation}
Con esta notaci\'on, el producto escalar de cuadrivectores puede expresarse como el producto escalar de los dos vectores de cuatro componente $a^\mu$ y $a_\mu$.
\subsection{Ejemplos de cuadrivectores}
%instiki:
\begin{align}
    x^\mu=&(x^0,x^1,x^2,x^3)=(t,x,y,z)=(t,\mathbf{x})\\
  p^\mu=&(p^0,p^1,p^2,p^3)=(E,p_x,p_y,p_z)=(E,\mathbf{p})
\end{align}
De la relatividad especial tenemos que
\begin{align}
  E=&\gamma m \nonumber\\
  \mathbf{p}=&\gamma m\mathbf{v}\,.
\end{align}
Por lo tanto, ya que $v^2=\mathbf{v}^2=|\mathbf{v}|^2$
\begin{align}
  E^2-\mathbf{p}^2=\gamma^2m^2(1-v^2)=m^2\,.
\end{align}
El invariante de Lorentz asociado a $p^\mu$ corresponde a la ecuaci\'on de momento energ\'\i a una vez se identifica la masa de una part\'\i cula con su cuadrimomentum
\begin{equation}
  p^2=p_\mu p^\mu=m^2=E^2-\mathbf{p}^2
\end{equation}
De \cite{uslhcblog}
\begin{quote}
  The intuitive understanding of this equation is that the energy of a particle is partially due to its motion and partially due to the intrinsic energy of its mass.  The application to particle detectors is that if you know the mass of a particular particle, or if it’s going so fast that its energy and momentum are both huge so that the mass can be roughly ignored, then knowing the energy tells you the momentum and vice versa
\end{quote}

Para $\mathbf{p}=0$, es decir cuando la part\'\i cula est\'a en reposo se reduce a la famosa ecuaci\'on.

Del electromagnetismo tenemos
\begin{equation}
  \label{eq:cv_jmu}
  J^\mu=(J^0,\mathbf{J})=(\rho,\mathbf{J})
\end{equation}
\begin{equation}
  \label{eq:cv_phia}
  A^\mu=(A^0,\mathbf{A})=(\phi,\mathbf{A})
\end{equation}
Del c\'alculo vectorial
\begin{align}
  \partial^\mu\equiv\frac{\partial}{\partial x_\mu}=&
  \left(
    \frac{\partial}{\partial x_0},\frac{\partial}{\partial x_1},\frac{\partial}{\partial x_2},\frac{\partial}{\partial x_3}
  \right)=\left(
    \frac{\partial}{\partial x^0},-\frac{\partial}{\partial x^1},-\frac{\partial}{\partial x^2},-\frac{\partial}{\partial x^3}
  \right)\nonumber\\
  =&\left(
    \frac{\partial}{\partial t},-\frac{\partial}{\partial x},-\frac{\partial}{\partial y},-\frac{\partial}{\partial z}
  \right)\nonumber\\
  =&(\partial_0,-\boldsymbol{\nabla})=(\partial^0,-\boldsymbol{\nabla})\\
  \partial_\mu=\frac{\partial}{\partial x^\mu}=&\left(
    \frac{\partial}{\partial t},\frac{\partial}{\partial x},\frac{\partial}{\partial y},\frac{\partial}{\partial z}
  \right)
  =(\partial_0,\boldsymbol{\nabla})
\end{align}
Por consiguiente:
\begin{equation}
  \label{eq:nabla}
  \boldsymbol{\nabla}=\frac{\partial}{\partial\mathbf{x}}
\end{equation}
Producto escalar:
\begin{equation}
  a_\mu b^\mu=g_{\mu\nu}a_\mu b^\nu=a^0b^0-a^1b^1-a^2b^2-a^3b^3=a^0b^0-a^i b^i=a^0b^0-\mathbf{a}\cdot \mathbf{b}
\end{equation}
Entonces
\begin{equation}
  \partial_\mu a^\mu=\frac{\partial a^0}{\partial t}+\boldsymbol{\nabla}\cdot\mathbf{a}
\end{equation}
La ecuaci\'on de continuidad $\partial_\mu J^\mu=0$ es un invariante de Lorentz.
El operador cuadr\'atico es, usando la ec.~\eqref{eq:146}
\begin{equation}
  \label{eq:dalambertian}
  \Box\equiv \partial_\mu\partial^\mu=\partial^0\partial^0-\nabla^2 =\frac{\partial^2}{\partial t^2}-\frac{\partial^2}{\partial x^2}-\frac{\partial^2}{\partial y^2}-\frac{\partial^2}{\partial z^2}=
\end{equation}
Por consiguiente la ecuaci\'on de onda en ec.~\eqref{eq:150} es invariante de Lorentz

Los operadores de energ\'\i a y momentum de la mec\'anica cu\'antica tambi\'en forma un cuadrivector
\begin{equation}
  \hat p^\mu=({\hat p}^0,\hat{\mathbf{p}})=(\widehat H,\hat{\mathbf{p}})
\end{equation}
con $\widehat H$, y $\hat{\mathbf{p}}$ dados en la ec.~\eqref{eq:151}. Entonces
\begin{equation}
  \label{eq:cv_hatpmu}
  \hat{p}^\mu=i\partial^\mu=i(\partial^0,\partial^i)=i(\frac{\partial}{\partial t},-\boldsymbol{\nabla})
\end{equation}
Del problema \ref{chap:tcc}.~\ref{item:pch1.3} se han definido las derivadas covariantes
\begin{align}
   \mathcal{D}_0=&\partial_0+i q A_0\nonumber\\
  \boldsymbol{\mathcal{D}}=&\boldsymbol{\nabla}-i q \mathbf{A}\nonumber\\
\end{align}
Podemos definir el cuadrivector
\begin{align}
  \mathcal{D}_\mu=&(\mathcal{D}_0,\boldsymbol{\mathcal{D}})\nonumber\\
=&(\partial_0,\boldsymbol{\nabla})+i q(A_0,-\mathbf{A})\nonumber\\
=&(\partial_0,\partial_i)+i q(A_0,-A^i)\nonumber\\
=&(\partial_0,\partial_i)+i q(A_0,A_i)\nonumber\\
=&(\partial_0+i q A_0,\partial_i+i A_i)\nonumber\\
=&(\mathcal{D}_0,\mathcal{D}_i)\nonumber\\
=&(\mathcal{D}_0,\mathcal{D}_i)\,,
\end{align}
donde hemos definido
\begin{align}
  \mathcal{D}_i=\partial_i+i q A_i\,.
\end{align}
Adem\'as $A^\mu$ tiene la transformaci\'on gauge
\begin{align}
  \mathbf{A}&\to\mathbf{A}'=\mathbf{A}+\boldsymbol{\nabla}\chi&
  A_0&\to A_0'=A_0-\frac{\partial\chi}{\partial t} 
\end{align}
En notaci\'on de cuadrivectores
\begin{align}
\label{eq:166qft}
  A^\mu\to {A'}^\mu=&\left(A^0-\frac{\partial\chi}{\partial t},\mathbf{A}+\boldsymbol{\nabla}\chi
  \right)\nonumber\\
  =&\left(A^0-\frac{\partial\chi}{\partial t},A^{i}+\partial_i\chi
  \right)\nonumber\\
  =&\left(A^0-\partial^0\chi,A^{i}-\partial^{i}\chi
  \right)\nonumber\\
  =&\left(A^0,A^{i}
  \right)-
  \left(
    \partial^0\chi,\partial^{i}\chi
  \right)\nonumber\\
  A^\mu\to {A'}^\mu=&A^\mu-\partial^\mu\chi\,.
\end{align}

\begin{english}
Note that the eq.~\eqref{eq:166qft} can be written as  
\end{english}
\begin{spanish}
Note que la ec.~\eqref{eq:166qft} se puede escribir como
\end{spanish}

\begin{align}
\label{eq:168qft}
  A_\mu\to A'_\mu=A_\mu-\partial_\mu\chi(x)
\end{align}
\begin{english}
which is just the transformation obtained in eq.~\eqref{eq:159qft}.  
\end{english}
\begin{spanish} %commments cannot use UTF8 characters
qu\'e es justamente la ecuaci\'on de transformaci\'on obtenida en la ec.~\eqref{eq:159qft}.  
\end{spanish}


\begin{example}
  Calcule la fracci\'on de la velocidad a la que puede ser acelerado un prot\'on en el LHC
Recuperando los factores de $c$
  \begin{align*}
  E=&\gamma m c^2&  \gamma=&\frac{1}{\sqrt{1-\beta^2}}
\end{align*}
\begin{align*}
  \beta=\frac{v}{c}=\sqrt{1-\frac{m^2 c^4}{E^2}}
\end{align*}
$m_p=938.272013(23) {\text{MeV}}/{c^2}$, and $E=7\,$TeV
\begin{align*}
  v=0.999999991\,c
\end{align*}
La longitud de un objeto esta definida tal que $t'=0$, de modo que $t=v x/c^2$, entonces
\begin{align}
  x'=\gamma(x-v t)=\gamma(x-v^2 x/c^2)=\sqrt{1-v^2/c^2}x.
\end{align}
Similarmente para la dilataci\'on temporal $x=0$ y
\begin{align}
  t'=\gamma t\,.
\end{align}
Por lo tanto observamos al prot\'on contra\'\i do en un factor de $1\times10^{-8}$
\end{example}

\begin{example}
  La amplitud de decaimiento del mu\'on es
    \begin{align}
      \Gamma_\mu=\left(\frac{G_F}{\sqrt{2}}\right)^2\frac{m_\mu^5}{96\pi^3}I\left(x\right)\,,
    \end{align}
    con $x=m_e/m_\mu$, e $I(x)=1-8x^2-24x^4\ln(x)+8x^6-x^8$.
    Entonces
    \begin{align}
      \Gamma_\mu=3.00867837568648 \times 10^{- 19} \; \text{GeV}
    \end{align}
    El tiempo de vida media del mu\'on se define como
    \begin{align}
      \tau_\mu=\frac{1}{\Gamma_\mu}=&3.32371850737231\times10^{18}\,\text{GeV}^{-1}\nonumber\\
      =&3.32371850737231\times10^{18}\times6.582\;118\;99\times10^{-25}\,\text{s}\nonumber\\
      =&2.197\;03(4)\times10^{-6}\,\text{s}\,.
    \end{align}
La longitud de decaimiento se define como
\begin{align}
  L_\mu=\frac{1}{\Gamma_\mu}=c\,\tau_\mu\approx658.65\,\text{m}\,.
\end{align}
El tiempo de vida media se refiere al tiempo de decaimiento para una part\'\i cula en reposo. Si $v=0.86\,c$, entonces
\begin{align}
  \tau_\mu'=\gamma\tau_\mu=\frac{\tau_\mu}{\sqrt{1-v^2}}\approx4.31\times10^{-6}\,\text{s}
\end{align}
el doble de cuando est\'a en reposo. 
\begin{align}
  L_\mu'=c\tau_\mu'=1290.74\,\text{m}\,.
\end{align}
A medida que el mu\'on se acerca m\'as a la velocidad de la luz, $L_\mu'$ coincide m\'as con la distancia recorrida por el mu\'on antes de decaer. De hecho se estima que despu\'es de ser producidos en la atm\'osfera de rayos c\'osmicos, a la superficie de la Tierra llegan unos 10000 muones por metro cuadrado cada minuto~\cite{muon}.
%examen: calcular la fracci\'on de la velocidad de la luz del muon para llegar a la superficie de la Tierra.
\end{example}




%\subsection{Transformaciones de Lorentz}
%\label{sec:transf-de-lorentz}

\subsection{Ecuaciones covariantes}
\label{sec:ecuac-covar}
Con el cuadrivector \eqref{eq:cv_hatpmu} podemos construir la
siguiente ecuaci\'on
\begin{align}
  \hat{p}_\mu\hat{p}^\mu\phi&=m^2\phi\nonumber\\
  i\partial_\mu i\partial^\mu\phi&=m^2\phi\nonumber\\
  -\partial_\mu\partial^\mu\phi&=m^2\phi\nonumber\\
  \label{eq:waveec}
  \left(\frac{\partial^2}{\partial t^2}-\nabla^2+m^2\right)\phi&=0.
\end{align}
Que corresponde a la ecuaci\'on de Klein-Gordon \eqref{eq:152}. Una expresi\'on escrita en t\'erminos de productos escalares de Lorentz se dice que esta en \emph{forma covariante}. El Lagrangiano covariante que da lugar a
\'esta ecuaci\'on es (ver ec. \eqref{eq:150}
\eqref{eq:15}). %noinstiki[in Cap. I](/wiki/show/Cap%C3%ADtulo+I#eq:15)). 

\begin{equation}
  \label{eq:wavelagtrue}
  \mathcal{L}=\frac{1}{2}\partial_\mu\phi\partial^\mu\phi-\frac{1}{2}m^2\phi^2
\end{equation}
El Lagrangiano m\'as general posible para el campo $\phi$ es en general bastante arbitrario:
\begin{equation}
  \mathcal{L}=\frac{1}{2}\partial_\mu\phi\partial^\mu\phi+f(\phi)\,,
\end{equation}
donde $f(\phi)$ es una función de campos escalar real $\phi$. Si $f(\phi)$ es una función polinómica del campo $\phi$, tenemos por ejemplo.
\begin{equation}
  \label{eq:wavelag}
  \mathcal{L}=\frac{1}{2}\partial_\mu\phi\partial^\mu\phi-\frac{1}{2}m^2\phi^2+\frac{1}{4}\lambda\phi^4+a\phi+b\phi^3.
\end{equation}
Un t\'ermino de la forma $\partial_\mu\partial^\mu\phi$ puede reabsorverse en la
ec.~\eqref{eq:wavelag} como una derivada total. Un posible t\'ermino
$J_\mu\partial^\mu\phi$, con $J_\mu$ constante, tambi\'en puede reescribirse como una
derivada total. Un t\'ermino constante
no afecta las ecuaciones de movimiento. Imponer la simetr\'\i a $\phi\to-\phi$
anula los dos \'ultimos t\'erminos. Potencias de $\phi$ mayores de cuatro
dar\'\i a lugar a una Teor\'\i a Cu\'antica de Campos no renormalizable. 

La dimensi\'on del campo $\phi$ puede obtenerse usando que la acci\'on es adimensional
\begin{equation}
  [S]\supset\left[\int d^4x\,m^2\phi^2\right]=E^{-4}E^2[\phi]^2\to [\phi]=E^1
\end{equation}
Diremos entonces que la dimensi\'on de $\phi$ es 1 (en unidades de energ\'\i a). Similarmente
\begin{equation}
  [S]\supset\left[\int d^4x\,\partial_\mu\phi\partial^\mu\phi\right]=E^{-4}[\partial_\mu]^2E^2\to [\partial_\mu]=E^1
\end{equation}
Como era de esperarse debido a que la derivada tiene unidades de longitud inversa.

Si hacemos $\lambda=a=b=0$ en la ec.~\eqref{eq:wavelag}, y usando las
ecuaciones de Euler-Lagrange \eqref{eq:eelcallfmu}, se obtiene
\begin{align}
  (\hat{p}_\mu\hat{p}^\mu-m^2)\phi&=0\nonumber\\
  \label{eq:k-gpmu} %noinstiki
(\hat{E}^2-\hat{\mathbf{P}}^2-m^2)\phi&=0\\
\label{eq:kg} 
  (\Box+m^2)\phi&=0,
\end{align}
donde
\begin{equation}
  \label{eq:dalambertiano}
  \Box\equiv\partial_\mu\partial^\mu=\frac{\partial^2}{\partial t^2}-\nabla^2. 
\end{equation}
Es el D'Alembartiano~\cite{daelembertiano}. 
Ec.~\eqref{eq:k-gpmu} %noinstikiEc.~\eqref{eq:kg}
corresponde a la forma de operadores de la
ecuaci\'on de energ\'\i a-momentum relativista. La
ec.~\eqref{eq:kg} se conoce como la ecuaci\'on de Klein-Gordon, con
Lagrangiano
\begin{equation}
  \label{eq:kglag}
  \mathcal{L}=\frac{1}{2}\partial_\mu\phi\partial^\mu\phi-\frac{1}{2}m^2\phi^2, 
\end{equation}
Una expresi\'on escrita en t\'erminos de productos escalares de
Lorentz se dice que esta en \emph{forma covariante}. Por lo tanto la
ecuaci\'on de Klein--Gordon y su correspondiente Lagrangiano est\'an
en forma covariante. Tambi\'en tienen la simetr\'\i a
$\phi\to-\phi$. A $\phi$ se le denomina \emph{campo escalar}.


\subsection{Lorentz tranformation for fields}

The scalar field is defined by their properties under Lorentz transformation. In section \ref{sec:teorema-de-noether} we study the behavior of one scalar field under a space--time translation. Under a general Lorentz transformation
\begin{align}
\label{eq:179qft}
  x^\mu\to {x'}^\mu={\Lambda^\mu}_\nu x^\nu\,,
\end{align}
Now we will study the effect of a Lorentz tranformation on the field $\phi(x)$, for example under a boost. By definition the scalar field does not change by the Lorentz transformation, the functional form is unaltered
the scalar field still satisfy
\begin{align}
  \phi(x)\to \phi'(x')=\phi(x)\,.
\end{align}
By using eq.~\eqref{eq:179qft} we have
\begin{align}
    \phi'(x')=\phi(\Lambda^{-1}x')\,.
\end{align}
Therefore, for an arbitrary space-time point we have that the scalar field transforms under a Lorentz transformation as
\begin{align}
  \label{eq:scalarlorentz}
   \phi(x)\to \phi'(x)=\phi(\Lambda^{-1}x)\,.
\end{align}

In order to check the Lorentz invariance of the scalar we need to obtain the Lorentz transformation properties for $\partial_\mu$. It is convinient to invert eq.~\eqref{eq:179qft}
\begin{align}
  {\left(\Lambda^{-1}\right)^\mu}_\alpha{x'}^\alpha=&{\left(\Lambda^{-1}\right)^\mu}_\alpha{\Lambda^\alpha}_\nu x^\nu\nonumber\\
=&\delta^\mu_\nu x^\nu\nonumber\\
=&x^\mu\,,
\end{align}
\begin{align}
  \frac{1}{{x'}^\nu}= {\left(\Lambda^{-1}\right)^\mu}_\nu\frac{1}{x^\mu}\,,
\end{align}
or
\begin{align}
  \label{eq:183qft}
    \frac{1}{{x'}^\mu}= {\left(\Lambda^{-1}\right)^\nu}_\mu\frac{1}{x^\nu}\,,
\end{align}
and the defintion of the Lorentz transformation itself:
\begin{align}
\label{eq:lrinvinv}
  g^{\mu\nu}={\left(\Lambda^{-1}\right)^\mu}_\rho\,g^{\rho\sigma}{\left(\Lambda^{-1}\right)^\nu}_\sigma\,.
\end{align}

From eq.~\eqref{eq:183qft} we can obtain the Lorentz transformation for $\partial_\mu=\partial/\partial x^\mu$:
\begin{align}
  \label{dmulrtran}
   \frac{\partial}{{\partial x'}^\mu}=& {\left(\Lambda^{-1}\right)^\nu}_\mu\frac{\partial}{\partial x^\nu}\nonumber\\
   {\partial\,}'_\mu=& {\left(\Lambda^{-1}\right)^\nu}_\mu\partial_\nu\,,
\end{align}
Podemos ahora demostrar que la Acción obtenida del Lagrangiano en la ec.\eqref{eq:kglag} es invariante bajo transformaciones de Lorentz
\begin{align}
  \mathcal{L}(x)\to  \mathcal{L}'(x)=& \frac{1}{2}\partial'_\mu\phi'\partial^{'\mu}\phi-\frac{1}{2}m^2\phi^{'2}, \nonumber\\
  =&{\left(\Lambda^{-1}\right)^\nu}_\mu g^{\mu \rho}{\left(\Lambda^{-1}\right)^\sigma}_\rho \partial_\nu\phi(\Lambda^{-1}x) \partial_\sigma \phi(\Lambda^{-1}x) -\frac{1}{2}m^2\phi^{2}(\Lambda^{-1}x)\nonumber\\
  =& g^{\nu \sigma}\partial_\nu\phi(\Lambda^{-1}x) \partial_\sigma \phi(\Lambda^{-1}x) -\frac{1}{2}m^2\phi^{2}(\Lambda^{-1}x)\nonumber\\
  =& \partial_\nu\phi(\Lambda^{-1}x) \partial^{\nu} \phi(\Lambda^{-1}x) -\frac{1}{2}m^2\phi^{2}(\Lambda^{-1}x)\nonumber\\
  =&\mathcal{L}(\Lambda^{-1}x)\,.
\end{align}
Since the Action involves the integration over all the points, it is invariant under the Lorentz transformation.




The field $A^\mu(x)$ transforms simultaneously as field and as vector under Lorentz transformation
\begin{align}
  A^\mu(x)\to {A'}^\mu(x')={\Lambda^\mu}_\nu A^\nu(\Lambda^{-1}x)\,.
\end{align}

\section{Campos escalares complejos}
Entre más simetrías posea una Acción menos arbitraría es. Podemos
ilustrar esta afirmación si consideramos una Acción para un campo escalar
complejo que además de ser invariante de Lorentz, se además invariante
bajo transformaciónes de fase.

En ese caso la Acción, y la correspondiente densidad Lagrnagian son
únicas y están dadas por una función polinómica de $\phi^{*}\phi$
\begin{align}
  \mathcal{L}=\partial_{\mu}\phi^{*} \partial^{\mu}\phi-m^2\phi^{*}\phi+\lambda \left(\phi^{*}\phi \right)^2\,.
\end{align}
Términos de orden superior se pueden obtener a partir de esa
Lagrangiana única y por eso no se consideran. 


De las ecuaciones de Euler-Lagrange para $\phi^*$, usando el Lagrangiano en ec.~(\ref{eq:41qft})
\begin{align}
  \partial_\mu\left[
      \frac{\partial\mathcal{L}}{\partial(\partial_\mu\phi^*)}\right]-\frac{\partial\mathcal{L}}{\partial\phi^*}&=0\nonumber\\
    \partial_\mu\partial^\mu\phi+m^2\phi&=0\nonumber\\
    \label{eq:43qft}
    (\Box+m^2)\phi&=0,
\end{align}
y de la ecuaciones de Euler-Lagrange para $\phi$,
\begin{equation}
  \label{eq:44qft}
    (\Box+m^2)\phi^*=0.
\end{equation}
De este modo tanto $\phi$, como $\phi^*$, satisfacen la ecuación de Klein-Gordon. Cada campo además corresponde a una partícula de masa $m$ como en el caso de $\phi_1$ y $\phi_2$

Estamos ahora interesado en las simetrías internas del Lagrangiano. Entonces la corriente conservada puede definida en la sección~\ref{sec:principio-de-minima-call}, eq.~\eqref{eq:jmuphi}
\begin{align}
  J^\mu=&\frac{\partial\mathcal{L}}{\partial(\partial_\mu\phi)}\delta\phi+\delta\phi^*\frac{\partial\mathcal{L}}{\partial(\partial_\mu\phi^*)}\nonumber\\
  \label{eq:45qft}
  J^\mu=&\partial^\mu\phi^*\delta\phi+\delta\phi^*\partial^\mu\phi.
\end{align}
Además de la invarianza de Lorentz, el Lagrangiano en ec,~(\ref{eq:41qft}) también es invariante bajo el grupo de transformaciones U(1) definido en las sección~\ref{sec:lagr-electr}, pero con una fase constante
\begin{equation*}
  U=e^{i\theta}\approx1+i\theta.
\end{equation*}
Entonces
\begin{align}
  \phi\overset{U}{\longrightarrow}\phi'&=e^{i\theta}\phi\approx(1+i\theta)\phi\nonumber\\
  &=\phi+i\theta\phi.
\end{align}
Entonces,
\begin{align}
  \delta\phi&=i\theta\phi\\
  \delta\phi^*&=-i\theta\phi^*.
\end{align}
Reemplazando en ec.~(\ref{eq:45qft})
\begin{equation}
\label{eq:46qft}
  J^\mu\propto -i\theta(\phi\partial^\mu\phi^*-\phi^*\partial^\mu\phi),
\end{equation}
y
\begin{equation}
\label{eq:47qft}
  \rho=J^0\propto-i\theta(\phi\frac{\partial\phi^*}{\partial t}-\phi^*\frac{\partial\phi}{\partial t}).
\end{equation}
Definimos $J^\mu$ como
\begin{equation}
  \label{eq:48qft}
   J^\mu= i(\phi^*\partial^\mu\phi-\phi\partial^\mu\phi^*),
\end{equation}
Como $\rho$ puede ser negativo no puede interpretarse como una
probalidad, como se hizo con la función de onda de la ecuación de
Scrödinger. Esto presentó un obstaculo en la interpretación inicial de
la ecuación de Klein-Gordon. Sin embargo una vez se cuantiza el
campo escalar la probabilidad de los estados cuánticos queda bien
definida \cite{Gross}. 



\section{Ecuaciones de Maxwell en notaci\'on covariante }
\label{sec:maxeqs}
Ecuaciones homog\'eneas:
\begin{align}
  \label{eq:hom_m_eq}
  \boldsymbol{\nabla}\cdot\mathbf{B}&=0,&\boldsymbol{\nabla}\times\mathbf{E}+\frac{\partial\mathbf{B}}{\partial t}&=0
\end{align}
Ecuaciones inhomog\'eneas:
\begin{align}
  \label{eq:inhom_m_eq}
  \boldsymbol{\nabla}\cdot\mathbf{E}&=\rho,&\boldsymbol{\nabla}\times\mathbf{B}-\frac{\partial\mathbf{E}}{\partial t}&=\mathbf{J}.
\end{align}
La primera ecuaci\'on establece la ausencia de cargas magn\'eticas, la segunda corresponde a la Ley de Faraday y la tercera a la Ley de Gauss. La cuarta sin el t\'ermino de desplazamiento el\'ectrico introducido por Maxwell corresponde a la Ley de Amp\`ere
\begin{equation}
   \boldsymbol{\nabla}\times\mathbf{B}=\mathbf{J}.
\end{equation}
Tomando la divergencia en esta expresi\'on tenemos
\begin{equation}
  \boldsymbol{\nabla}\cdot\mathbf{J}=0,
\end{equation}
que corresponde a la ecuaci\'on de continuidad \eqref{eq:conti} para $\rho$ constante
\begin{equation}
  \label{eq:153}
  \frac{\partial \rho}{\partial t}+\boldsymbol{\nabla}\cdot\mathbf{J}=0.
\end{equation}
De este modo la Ley de Amp\`ere da lugar a la conservaci\'on de carga el\'ectrica pero solo a nivel global:  una perdida de carga el\'ectrica en un punto del universo puede ser compensada por la aparici\'on instant\'anea de carga el\'ectrica en otro lugar del universo. La conservaci\'on global podr\'\i a necesitar la propagaci\'on instant\'anea de se\~nales, y esto est\'a en conflicto con la relatividad especial.


Tomando la divergencia de la Ley de Amp\`ere modificada por Maxwell
\begin{equation}
   \boldsymbol{\nabla}\times\mathbf{B}-\frac{\partial\mathbf{E}}{\partial t}=\mathbf{J},
\end{equation}
obtenemos la ecuaci\'on de continuidad \eqref{eq:153}. Dicha ecuaci\'on establece que la raz\'on de decrecimiento de la carga en un volumen arbitrario $V$ es debido precisa y \'unicamente al flujo de la corriente fuera de su superficie; de modo que la carga no puede ser creada ni destruida dentro de $V$.  Ya que $V$ puede ser arbitrariamente peque\~no esto significa que la carga el\'ectrica debe conservarse localmente.   El t\'ermino extra introducido por Maxwell est\'a motivado por un requerimiento de conservaci\'on local. 

A la luz del teorema de Noether la conservaci\'on local de la carga el\'ectrica debe provenir de una transformaci\'on continua y \emph{local} que deje invariante a la ecuaciones de Maxwell. Las invarianza gauge de la ecuaciones de Maxwell juegan este papel. Las cargas conservadas localmente pueden determinarse a partir de la din\'amica del sistema  \cite{Aitchison:2003tq}, adem\'as del uso de cargas conocidas que participen en alguna reacci\'on. Por ejemplo se puede estudiar la forma como responde una part\'\i cula de carga desconocida a campos electromagn\'eticos para determinar su carga. 

El principio gauge local, que pretendemos formular, va m\'as all\'a del teorema de Noether estableciendo una relaci\'on entre las simetr\'\i as, las leyes de conservaci\'on y la din\'amica. Este se constituye en el paradigma para estudiar las interacciones relevantes en f\'\i sica de part\'\i culas.

Para mostrar la invarianza gauge que exhiben las ecuaciones de Maxwell, es conveniente reescribirlas en forma covariante. Para ello es conveniente usar el potencial escalar el\'ectrico $\phi$ y el potencia vectorial magn\'etico $\mathbf{A}$.

%to_en:The following equations are equivalent to the two homogenous Maxwell equations
Las siguientes ecuaciones son equivalentes a las ecuaciones homog\'eneas de Maxwell
\begin{align}
  \label{eq:phia}
  \mathbf{E}&=-\boldsymbol{\nabla}\phi-\frac{\partial\mathbf{A}}{\partial t}&
  \mathbf{B}&=\boldsymbol{\nabla}\times\mathbf{A}.
\end{align}
Ya que
\begin{align*}
  \boldsymbol{\nabla}\times\mathbf{E}&=-\boldsymbol{\nabla}\times\boldsymbol{\nabla}\phi-\frac{\partial}{\partial t}\boldsymbol{\nabla}\times\mathbf{A}\\
  &=-\frac{\partial\mathbf{B}}{\partial t},
\end{align*}
y
\begin{align*}
  \boldsymbol{\nabla}\cdot\mathbf{B}&=\boldsymbol{\nabla}\cdot(\boldsymbol{\nabla}\times\mathbf{A})\\
  &=0
\end{align*}
Usando el cuadrivector en ec.~(\ref{eq:cv_phia}) y
expandiendo la ec.~(\ref{eq:phia}), tenemos
\begin{align}
  E^i&=-(\frac{\partial A^0}{\partial x^{i}}+\frac{\partial A^{i}}{\partial x^0})\nonumber\\
  &=(\frac{\partial A^0}{\partial x_i}-\frac{\partial A^{i}}{\partial x_0})\nonumber\\
  &=\partial^{i}A^0-\partial^0 A^{i}\nonumber\\
  \label{eq:Efmunu}
  &=\partial^\mu A^\nu-\partial^\nu A^\mu,
  \qquad 
  \mu=i,\quad \nu=0\\
  \label{eq:E_Fi0} %noinstiki
  E^{i}&=F^{i0}
\end{align}
donde hemos definido el Tensor de intensidad electrom\'agnetica como:
\begin{equation}
  \label{eq:fmunu}
    F^{\mu\nu}=\partial^\mu A^\nu-\partial^\nu A^\mu.
\end{equation}
A $A^\mu$ se le denomina \emph{campo vectorial}. Similarmente
\begin{align}
  B^k&=\epsilon_{ijk}\frac{\partial A^j}{\partial x^{i}}\nonumber\\
  \epsilon_{lmk}B^k&=\epsilon_{lmk}\epsilon_{ijk}\frac{\partial A^j}{\partial x^{i}}\nonumber\\
  &=(\delta_{li}\delta_{mj}-\delta_{lj}\delta_{mi})\frac{\partial A^j}{\partial x^{i}}\nonumber\\
  &=\frac{\partial A^m}{\partial x^l}-\frac{\partial A^l}{\partial x^m}\nonumber\\
  &=-\frac{\partial A^m}{\partial x_l}+\frac{\partial A^l}{\partial x_m}\nonumber\\
  &=\partial^m A^l-\partial^l A^m\nonumber\\
  \label{eq:Bfmunu}
  &=\partial^\mu A^\nu-\partial^\nu A^\mu,
  \qquad 
  \mu=m,\quad \nu=l.\\
  \label{eq:BFij}
\epsilon_{lmk}B^k&=F^{ml}
\end{align}
Por consiguiente la ec.~(\ref{eq:fmunu}) es tambi\'en equivalente a las
dos ecuaciones homog\'eneas de Maxwell. En forma matricial,
\begin{align}
  F^{\mu\nu}&=
  \begin{pmatrix}
    0       &\partial^0A^1-\partial^1A^0&\partial^0A^2-\partial^2A^0&\partial^0A^3-\partial^3A^0\\
    \partial^1A^0-\partial^0A^1&0       &\partial^1A^2-\partial^2A^1&\partial^1A^3-\partial^3A^1\\
    \partial^2A^0-\partial^0A^2&\partial^2A^1-\partial^1A^2&0       &\partial^2A^3-\partial^3A^2\\
    \partial^3A^0-\partial^0A^3&\partial^3A^1-\partial^1A^3&\partial^3A^2-\partial^2A^3&0\\
  \end{pmatrix}\nonumber\\
  &=
  \begin{pmatrix}
    0 &-E^1   &-E^2   &-E^3   \\    
    E^1&0     &\epsilon_{213}B^3&\epsilon_{312}B^2\\
    E^2&\epsilon_{123}B^3&0     &\epsilon_{321}B^1\\
    E^3&\epsilon_{132}B^2&\epsilon_{231}B^1&0\\
  \end{pmatrix}\nonumber\\
  &=
\label{eq:matrixfmunu}
  \begin{pmatrix}
    0 &-E^1&-E^2&-E^3   \\    
    E^1&0  &-B^3&B^2\\
    E^2&B^3 &0  &-B^1\\
    E^3&-B^2&B^1 &0\\
  \end{pmatrix}.
\end{align}

La ec.~(\ref{eq:fmunu}) satisface la identidad,
\begin{equation}
  \label{eq:homME3}
  \partial^\lambda F^{\mu\nu}+\partial^\mu F^{\nu\lambda}+\partial^\nu F^{\lambda\mu}=0
\end{equation}
Definiendo el tensor dual como
\begin{equation*}
  \tilde{F}^{\mu\nu}=\frac{1}{2}\epsilon^{\mu\nu\rho\sigma}F_{\rho\sigma},
\end{equation*}
la ec.~(\ref{eq:homME3}) puede escribirse como
\begin{equation}
  \label{eq:homEM4}
  \partial_\mu\tilde{F}^{\mu\nu}=0.
\end{equation}

Para reescribir las ecuaciones de Maxwell inhomog\'enas en forma
covariante usaremos adem\'as el cuadrivector $J^\mu$ de la
ec.~\eqref{eq:cv_jmu}.  Usando la ec.~\eqref{eq:Efmunu}, la primera
ecuaci\'on de Maxwell inhomog\'enea~\eqref{eq:inhom_m_eq} puede escribirse
como
\begin{align}
  \label{nohomME21}
    \frac{\partial E^{i}}{\partial x^{i}}&=J^0\nonumber\\
    \frac{\partial}{\partial x^{i}}F^{i0}&=J^0\nonumber\\
    \partial_iF^{i0}&=J^0\nonumber\\
    \partial_\mu F^{\mu0}&=J^0\,.
\end{align}
Usando las ecs.~\eqref{eq:Efmunu}, \eqref{eq:Bfmunu}, la segunda
ecuaci\'on de Maxwell inhomog\'enea~\eqref{eq:inhom_m_eq} puede escribirse
como
\begin{align}
  \epsilon_{ijk}\frac{\partial B^j}{\partial x^{i}}-\frac{\partial E^k}{\partial t}&=J^k\nonumber\\
  -\frac{\partial (\epsilon_{ikj}B^j)}{\partial x^{i}}-\frac{\partial E^k}{\partial t}&=J^k\nonumber\\
-\partial_iF^{ki}-\partial_0F^{k0}&=J^k\nonumber\\
\partial_iF^{ik}+\partial_0F^{0k}&=J^k\nonumber\\
\label{nohomME22}
\partial_\mu F^{\mu k}&=J^k
\end{align}

Las ecuaciones \eqref{nohomME21},
\eqref{nohomME22} pueden escribirse en forma compacta como
\begin{align}
\label{eq:nohomME2}
  \partial_\mu F^{\mu\nu}&=J^\nu\\
&=
\begin{cases}
  \partial_\mu(\partial^\mu A^0-\partial^0A^{i})=J^0&\text{para $\nu=0$}\\
  \partial_\mu(\partial^\mu A^{i}-\partial^{i}A^\mu)=J^{i}&\text{para $\nu=i$}
\end{cases}\nonumber
\end{align}


Los resultados sobre la notaci\'on covariante de la ecuaciones de Maxwell est\'an resumidos en la Tabla~\ref{tab:eqmax} %noinstiki
\begin{table} %noinstiki
  \begin{center} %noinstiki
  \begin{tabular}{l|l|l|l} %noinstiki
            &$\mathbf{E}$, $\mathbf{B}$&$A^\mu$            &$F^{\mu\nu}$\\\hline %noinstiki
Homog\'eneas  &Ec.~(\ref{eq:hom_m_eq})   &~(\ref{eq:phia})&~(\ref{eq:fmunu}) \'o (\ref{eq:homME3}) \'o (\ref{eq:homEM4})\\\hline %noinstiki
Inhomog\'eneas& (\ref{eq:inhom_m_eq})    &                &~\eqref{eq:nohomME2} \\ %noinstiki
  \end{tabular} %noinstiki
  \end{center} %noinstiki
  \caption{Ecuaciones de Maxwell} %noinstiki
  \label{tab:eqmax} %noinstiki
\end{table} %noinstiki
De la parte izquierda de la ecuaci\'on \eqref{eq:nohomME2}, podemos ver que
\begin{align*}
  \partial_\nu\partial_\mu F^{\mu\nu}&=0
\end{align*}
Por consiguiente, la cuadricorriente $J^\mu$ es conservada:
\begin{equation}
  \label{eq:consvjmu}
  \partial_\mu J^\mu=0
\end{equation}


\subsection{Lagrangiano Electromagn\'etico}
\label{sec:lagr-electr}
%to_en:Eq.~(\ref{eq:phia}) is invariant under the following transformations
La ec.~(\ref{eq:phia}) es invariante bajo las siguientes transformaciones
\begin{align}
  \label{eq:phia_transf}
  \mathbf{A}&\to\mathbf{A}'=\mathbf{A}+\boldsymbol{\nabla}\chi&
  \phi&\to\phi'=\phi-\frac{\partial\chi}{\partial t} 
\end{align}
%to_en:Since
Ya que
\begin{equation}
  \label{eq:Etrans}
  \mathbf{E}\to\mathbf{E}'= -\boldsymbol{\nabla}\phi+\frac{\partial}{\partial t}\boldsymbol{\nabla}\chi
  -\frac{\partial\mathbf{A}}{\partial t}-\frac{\partial}{\partial t}\boldsymbol{\nabla}\chi=\mathbf{E}
\end{equation}
\begin{equation}
  \label{eq:btransf}
  \mathbf{B}\to\mathbf{B}'= \boldsymbol{\nabla}\times\mathbf{A}+
  \underbrace{\boldsymbol{\nabla}\times\boldsymbol{\nabla}\chi}_{\displaystyle =0}=\mathbf{B}
\end{equation}
%to_en:This imply that different observers in different space points, using different calibrations for their measures, get the same fields. 
Esto implica que diferentes observadores en diferentes puntos del espacio, usando diferentes calibraciones para sus medidas, obtienen los mismos campos. Las  ecs.~\eqref{eq:phia_transf}, corresponden a \emph{transformaciones gauge locales}

%to_en:In covariant notation
En notaci\'on de cuadrivectores
\begin{align}
  A^\mu\to {A'}^\mu=&\left(\phi-\frac{\partial\chi}{\partial t},\mathbf{A}+\boldsymbol{\nabla}\chi
  \right)\nonumber\\
  =&\left(\phi-\frac{\partial\chi}{\partial t},A^{i}+\partial_i\chi
  \right)\nonumber\\
  =&\left(\phi-\partial^0\chi,A^{i}-\partial^{i}\chi
  \right)\nonumber\\
  =&\left(\phi,A^{i}
  \right)-
  \left(
    \partial^0\chi,\partial^{i}\chi
  \right)\nonumber\\
  \label{eq:aphicov}
  A^\mu\to {A'}^\mu=&A^\mu-\partial^\mu\chi
\end{align}
%to_en: Let $U$ an element of the Transformation Group $U(1)$:
Sea  $U$ un elemento del Grupo de Transformaciones  $U(1)$:
\begin{equation}
  \label{eq:u1ele}
  U=e^{i\theta(x)}\in U(1)
\end{equation}
%to_en: The Group is defined by the infinity set of elements $U_i=e^{i\theta(x_i)}$. Then
El Grupo est\'a definido por el conjunto infinito de elementos $U_i=e^{i\theta(x_i)}$. Entonces
\begin{itemize} %noinstiki
%instiki:
\item Producto de Grupo 
  \begin{equation*}
      U_1\cdot U_2=e^{i[\theta(x_1)+\theta(x_2)]}\equiv e^{i\theta(x_3)}\in U(1)
  \end{equation*}
\item Identidad: 
  \begin{equation*}
  \theta(x)=0\qquad \text{tal que}\qquad U_I=1  
  \end{equation*}
\item Inverso 
  \begin{equation*}
      \theta(-x)=-\theta(x)\qquad \text{tal que}\qquad U^{-1}=e^{-i\theta(x)}
  \end{equation*}
\end{itemize} %noinstiki
Note que si
\begin{equation}
  \label{eq:amutransf}
  A^\mu\to{A'}^\mu=A^\mu-\frac{i}{e}(\partial^\mu U)U^{-1}
\end{equation}
%to_en:(If $\theta(x)=$cte, $ A^\mu={A^\mu}'$, phase invariance). If $\theta$ is sufficiently small
(Si $\theta=$cte, $ A^\mu={A'}^\mu$, invarianza de fase). Si $\theta$ es suficientemente peque\~no
\begin{align}
  \label{eq:Uinf}
  U=e^{i\theta(x)}&\approx1+i\theta(x)+\mathcal{O}(\theta^2)&U^{-1}=e^{-i\theta(x)}&\approx1-i\theta(x)+\mathcal{O}(\theta^2)
\end{align}
Entonces
\begin{align}
  \label{eq:checkinft}
  {A^\mu}'=&-\frac{1}{e}(i\partial^\mu\theta(8x))[1+i\theta(x)+\mathcal{O}(\theta^2)][1-i\theta(x)+\mathcal{O}(\theta^2)]
  +A^\mu[1+i\theta(x)+\mathcal{O}(\theta^2)][1-i\theta(x)+\mathcal{O}(\theta^2)]\nonumber\\
  =&A^\mu+\frac{1}{e}\partial^\mu\theta(x)+\mathcal{O}(\theta^2)
\end{align}
%to_en:Therefore, if $\chi(x)=-(1/e)\theta(x)$, then eq.~(\ref{eq:aphicov}) is the infinitesimal version of the $U(1)$ transformation in eq.~(\ref{eq:amutransf})
Por consiguiente, si $\chi(x)=-(1/e)\theta(x)$, entonces la ec.~(\ref{eq:aphicov}) es la versi\'on infinitesimal de la transformaci\'on $U(1)$  en ec.~(\ref{eq:amutransf}). Del Teorema de Noether debe existir una carga conservada corresponde a la carga el\'ectrica, asociada la corriente $J^\mu$, de la cual a\'un no hemos especificado su origen. 

%to_en:Under this transformation
Bajo esta transformaci\'on
\begin{align}
  \label{eq:fmunutrans}
  F^{\mu\nu}\to{F'}^{\mu\nu}=&(\partial^\mu{A'}^\nu-\partial^\nu{A'}^\mu)\nonumber\\
  =&\partial^\mu A^\nu-\partial^\mu\partial^\nu\chi-\partial^\nu A^\mu+\partial^\nu\partial^\mu\chi\nonumber\\
  =&\partial^\mu A^\nu-\partial^\nu A^\mu-\partial^\mu\partial^\nu\chi+\partial^\mu\partial^\nu\chi\nonumber\\
  =&F^{\mu\nu}
\end{align}
%to_en:In this way the homogeneous part of Maxwell equations corresponds to a Gauge Theory!. This was a curiosity until 1950's. 
De este modo las ecuaciones homog\'eneas de Maxwell \eqref{eq:fmunu} son invariantes gauge. Como la transformaci\'on gauge solo afecta al campo $A^\mu$, las ecuaciones inhomog\'eneas de Maxwell \eqref{eq:nohomME2} tambi\'en son invariantes gauge. 
De esta forma las ecuaciones de Maxwell corresponde a una Teor\'\i a Gauge!. Esto fue una curiosidad hasta los 1950. 

Para ilustrar la relaci\'on entre la conservaci\'on local de la carga el\'ectrica y la transformaci\'on gauge, que no es una conexi\'on simple, considere las ecuaciones de Maxwell
\begin{equation}
    \partial_\mu F^{\mu\nu}=J^\nu,
\end{equation}
que autom\'aticamente incluyen la conservaci\'on local de carga, expresada por la ecuaci\'on de continuidad
\begin{equation}
  \partial_\mu J^\mu=0.  
\end{equation}
Adem\'as las ecuaciones de Maxwell permanecen invariantes bajo la transformaci\'on gauge
\begin{equation}
   A^\mu\to {A'}^\mu=A^\mu-\partial^\mu\chi,
\end{equation}
ya que dicha transformaci\'on deja invariante a $F^{\mu\nu}$. De aqu\'\i{} la sugerencia de la invarianza gauge esta relacionada de alguna manera a la conservaci\'on de la carga. De hecho la acci\'on m\'as general posible para el campo $A^\mu$ compatible tanto con la invarianza de Lorentz y la invarianza gauge local corresponde a la acci\'on que da lugar a las ecuaciones de Maxwell.



\section{Vector field Lagrangian}

We are now are in position to answer the following question: What is the most general Lagrangian for a the four--components field $A^\mu$ compatible with Lorentz invariance and the gauge transformation
\begin{align}
\label{eq:172qft}
  A^\mu\to{A'}^\mu=A^\mu-\partial^\mu\chi(x)\;?
\end{align}




Definiendo
\begin{align*}
  F^{\mu\nu}&=\partial^\mu A^\nu-\partial^\nu A^\mu\\
  G^{\mu\nu}&=\partial^\mu A^\nu+\partial^\nu A^\mu\\
\end{align*}
El Lagrangiano que da lugar a una Acción invariante de Lorentz para el cuadrivector $A^\mu$
es, hasta derivadas totales y potencias en los campos de hasta dimensión 4:
\begin{align}
  \mathcal{L}=&-\frac{1}{4}F^{\mu\nu}F_{\mu\nu}-\frac{1}{4}G^{\mu\nu}G_{\mu\nu}-J^\mu A_\mu+
 \frac{1}{2}m^2A^\mu A_\mu +\lambda_1\partial_\nu A^\nu(x) A_\mu(x) A^\mu(x)+\lambda_2 A^\mu A_\mu A^\nu A_\nu\nonumber\\
&+\lambda_3 F^{\mu\nu}(x)A_\mu(x) A_\nu(x)+\lambda_4G^{\mu\nu}(x)A_\mu(x) A_\nu\,,\,.
  \label{eq:lagAmu}
\end{align}
\begin{itemize}
\item \textbf{Ejercicio:} Show that terms like $\partial^\mu A^\nu(x)\partial_\mu A_\nu(x)$, and hence $F^{\mu\nu}F_{\mu\nu}$, transforms as
  \begin{align}
    \partial^\mu A^\nu\left(\Lambda^{-1}x\right)\partial_\mu A_\nu\left(\Lambda^{-1}x\right)
  \end{align}
Hint: use the Lorentz transformation properties of $\partial_\mu$ in eq.~\eqref{dmulrtran}.
\end{itemize}
In the case of $J^\mu A_\mu$:
\begin{align}
  J^\mu(x)A_\mu(x)\to g_{\mu\nu}{J'}^\mu(x){A'}^\nu(x)=& g_{\mu\nu}{\Lambda^\mu}_\rho J^\rho\left(\Lambda^{-1}x\right){\Lambda^\nu}_\sigma A^\sigma\left(\Lambda^{-1}x\right)\nonumber\\
=& {\Lambda^\mu}_\rho g_{\mu\nu}{\Lambda^\nu}_\sigma J^\rho\left(\Lambda^{-1}x\right)A^\sigma\left(\Lambda^{-1}x\right)\nonumber\\
=& g_{\rho\sigma}J^\rho\left(\Lambda^{-1}x\right)A^\sigma\left(\Lambda^{-1}x\right)\,,
\end{align}
in the case $\partial_\nu A^\nu(x) A_\mu(x) A^\mu(x)$:
\begin{align}
   \partial_\nu A^\nu(x) A_\mu(x) A^\mu(x)\to {\partial'}_\nu{A'}^\nu(x') {A'}_\mu(x') {A'}^\mu(x')=& {\left(\Lambda^{-1}\right)^\sigma}_\nu{\Lambda^\nu}_\rho\partial_\sigma A^\rho\left(\Lambda^{-1}x\right) A_\mu\left(\Lambda^{-1}x\right) A^\mu\left(\Lambda^{-1}x\right)\nonumber\\
=& \delta^\sigma_\rho\partial_\sigma A^\rho\left(\Lambda^{-1}x\right) A_\mu\left(\Lambda^{-1}x\right) A^\mu\left(\Lambda^{-1}x\right)\nonumber\\
=& \partial_\rho A^\rho\left(\Lambda^{-1}x\right) A_\mu\left(\Lambda^{-1}x\right) A^\mu\left(\Lambda^{-1}x\right)\,,\nonumber\\
\end{align}
and similarly for the other terms. Under a Lorentz transformation the full Lagrangian transform as
\begin{align}
  \mathcal{L}(x)\to\mathcal{L}'(x)=\mathcal{L}(\Lambda^{-1}x) 
\end{align}
Since the Action involves the integration over all the points, it is invariant under the Lorentz transformation. The $J^\mu(x)$ does not involves the introduction a new vector field, because it will be identified later as the 4--current.


Terms like
\begin{align}
  K_\nu A^\nu(x) A_\mu(x) A^\mu(x)\,,
\end{align}
(for $K_\nu$ constant) are not Lorentz invariant:
\begin{align}
  K_\nu A^\nu(x) A_\mu(x) A^\mu(x)\to K_\nu{A'}^\nu(x) {A'}_\mu(x) {A'}^\mu(x)=& K_\nu{\Lambda^\nu}_\rho A^\rho\left(\Lambda^{-1}x\right) A_\mu\left(\Lambda^{-1}x\right) A^\mu\left(\Lambda^{-1}x\right)\,.
\end{align}
$K_\nu(x)A^\nu(x)A_\mu(x)A^\mu(x)$ is Lorentz covariant but not gauge-invariant (see below).

%to_en:Under this transformation
Bajo la transformación gauge \eqref{eq:168qft}
\begin{align}
  \label{eq:fmunutrans}
  F^{\mu\nu}\to{F'}^{\mu\nu}=&(\partial^\mu{A'}^\nu-\partial^\nu{A'}^\mu)\nonumber\\
  =&\partial^\mu A^\nu-\partial^\mu\partial^\nu\chi-\partial^\nu A^\mu+\partial^\nu\partial^\mu\chi\nonumber\\
  =&\partial^\mu A^\nu-\partial^\nu A^\mu-\partial^\mu\partial^\nu\chi+\partial^\mu\partial^\nu\chi\nonumber\\
  =&F^{\mu\nu}
\end{align}

Si queremos que la Acción refleja las simetrías de las ecuaciones de
Maxwell debemos mantener sólo los términos del Lagrangiano para $A^\mu$
en \eqref{eq:lagAmu} que sean invariantes hasta una derivada total. Bajo una transformación gauge, cada
uno de los términos
\begin{equation*}
  -\frac{1}{4}G^{\mu\nu}G_{\mu\nu}+
  \frac{1}{2}m^2A^\mu A_\mu+\lambda_1\partial_\mu A^\mu A_\nu A^\nu+\lambda_2 A^\mu A_\mu A^\nu A_\nu+\lambda_3F^{\mu\nu}A_\mu A_\nu+\lambda_4G^{\mu\nu}A_\mu A_\nu
+K_\nu(x) A^\nu A_\mu A^\mu
\end{equation*}
dan lugar a un $\delta\mathcal{L}\neq\partial_\mu(\text{algo})$ y la Acción no es
invariante bajo la transformación gauge. Para los 
términos restantes
\begin{align}
    \mathcal{L}=-\frac{1}{4}F^{\mu\nu}F_{\mu\nu}-J^\mu A_\mu\,,
\end{align}

 usando la ec.~\eqref{eq:consvjmu}, tenemos
\begin{align}
  \delta\mathcal{L}=\mathcal{L}'-\mathcal{L}=&
-\frac{1}{4}{F'}^{\mu\nu}F'_{\mu\nu}-J^\mu A'_\mu+\frac{1}{4}F^{\mu\nu}F_{\mu\nu}+J^\mu A_\mu\nonumber\\
=&-J^\mu A_\mu+J^\mu\partial_\mu\theta(x)-J^\mu A_\mu\nonumber\\
=&\partial_\mu(J^\mu\chi)-(\partial_\mu J^\mu)\theta(x)
\end{align}
For the action
\begin{align}
  \delta S=&\int d^4x \left[\partial_\mu(J^\mu\chi)-(\partial_\mu J^\mu)\theta(x)\right]\nonumber\\
=&-\int d^4x (\partial_\mu J^\mu)\theta(x)\nonumber\\
=&-\int d^3x\int_{-\infty}^\infty dt (\partial_\mu J^\mu)\theta(x)\,.
\end{align}
In order to have $\delta S=0$ we need to assume for the while that $\partial_\mu J^\mu=0$. However we will see that this is just a self-consistent condition.

In summary, if the electromagnetic current is conserved, then the Lagrangian is invariant under the gauge transformation \eqref{eq:172qft}. Note that the Lagrangian density is not locally gauge invariant. However, the action (and hence the theory) is gauge invariant.

Por lo tanto, el Lagrangiano
\begin{equation}
  \label{eq:lagAmum}
  \mathcal{L}=-\frac{1}{4}F^{\mu\nu}F_{\mu\nu}-J^\mu A_\mu
\end{equation}
es el más general que da lugar a una Acción invariante de Lorentz e invariante gauge
local. 



The definition of $F^{\mu\nu}$ already includes the homogeneous Maxwell equations. To see this we note first that the only non-zero $F^{\mu\nu}$ components are
\begin{align}
  F^{\mu\nu}=  \begin{cases}
    F^{\mu0}=F^{i0} & \text{$\nu=0$}\\
    F^{\mu l}=F^{ml} & \text{$\nu=l$}\\
  \end{cases}
\end{align}
For $\nu=0$ we have
\begin{align}
F^{i0}  &=\partial^{i}A^0-\partial^0 A^{i}\nonumber\\
  &=(\frac{\partial A^0}{\partial x_i}-\frac{\partial A^{i}}{\partial x_0})\nonumber\\
&=-(\frac{\partial A^0}{\partial x^{i}}+\frac{\partial A^{i}}{\partial x^0})\nonumber\\
&=E^i
\end{align}
where
\begin{align}
\label{eq:173qft}
   \mathbf{E}&=-\boldsymbol{\nabla}\phi-\frac{\partial\mathbf{A}}{\partial t}\,.
\end{align}
while for $\nu=l$ we have
\begin{align}
F^{ml}  &=\partial^m A^l-\partial^l A^m\nonumber\\
  &=(\delta_{lj}\delta_{mi}-\delta_{li}\delta_{mj}){\partial^iA^j}\nonumber\\
  &=-(\delta_{lj}\delta_{mi}-\delta_{li}\delta_{mj}){\partial_iA^j}\nonumber\\
  &=(\delta_{li}\delta_{mj}-\delta_{lj}\delta_{mi}){\partial_iA^j}\nonumber\\
  &=(\delta_{li}\delta_{mj}-\delta_{lj}\delta_{mi})\frac{\partial A^j}{\partial x^{i}}\nonumber\\
    &=\epsilon_{lmk}\epsilon_{ijk}\frac{\partial A^j}{\partial x^{i}}\nonumber\\
&=\epsilon_{lmk}\left(\boldsymbol{\nabla}\times  \mathbf{A}\right)^k\nonumber\\
&=\epsilon_{lmk}B^k\,,
\end{align}
where
\begin{align}
\label{eq:174qft}
  \mathbf{B}&=\boldsymbol{\nabla}\times \mathbf{A}\,.
\end{align}
Then we have
\begin{align}
  \{F^{\mu\nu}\}  &=
  \begin{pmatrix}
    0 &-E^1   &-E^2   &-E^3   \\    
    E^1&0     &\epsilon_{213}B^3&\epsilon_{312}B^2\\
    E^2&\epsilon_{123}B^3&0     &\epsilon_{321}B^1\\
    E^3&\epsilon_{132}B^2&\epsilon_{231}B^1&0\\
  \end{pmatrix}\nonumber\\
  &=
\label{eq:matrixfmunu}
  \begin{pmatrix}
    0 &-E^1&-E^2&-E^3   \\    
    E^1&0  &-B^3&B^2\\
    E^2&B^3 &0  &-B^1\\
    E^3&-B^2&B^1 &0\\
  \end{pmatrix}.
\end{align}


From eqs.~\eqref{eq:173qft}, and \eqref{eq:174qft}
\begin{align*}
  \boldsymbol{\nabla}\times \mathbf{E}&=-\boldsymbol{\nabla}\times \boldsymbol{\nabla}\phi-\frac{\partial}{\partial t}\boldsymbol{\nabla}\times \mathbf{A}\\
  &=-\frac{\partial\mathbf{B}}{\partial t},
\end{align*}
and
\begin{align*}
  \boldsymbol{\nabla}\cdot\mathbf{B}&=\boldsymbol{\nabla}\cdot(\boldsymbol{\nabla}\times \mathbf{A})\\
  &=0
\end{align*}
which are just the homogeneous Maxwell equations. Therefore the expression
 \begin{equation}
  \label{eq:fmunu}
    F^{\mu\nu}=\partial^\mu A^\nu-\partial^\nu A^\mu.
\end{equation}
with the $\{F^{\mu\nu}\}$ given in \eqref{eq:matrixfmunu}, is just an equivalent form for the homogeneous Maxwell equations. The remaining Maxwell equations can be obtained from the Euler-Lagrange equations for $A^\nu$:


Con miras a calcular  las ecuaciones de Euler-Lagrange para el Lagrangiano en
ec.~\eqref{eq:lagAmum}, tenemos
\begin{align}
F^{\rho\sigma}F_{\rho\sigma}=&(\partial^\rho A^\sigma-\partial^\sigma A^\rho)(\partial_\rho A_\sigma-\partial_\sigma A_\rho)\nonumber\\
=&\partial^\rho A^\sigma\partial_\rho A_\sigma-\partial^\rho A^\sigma\partial_\sigma A_\rho-\partial^\sigma A^\rho\partial_\rho A_\sigma+\partial^\sigma A^\rho\partial_\sigma
A_\rho\nonumber\\
=&g^{\rho\alpha}g^{\sigma\beta}(\partial_\alpha A_\beta\partial_\rho A_\sigma-\partial_\alpha A_\beta\partial_\sigma A_\rho-\partial_\beta A_\alpha\partial_\rho A_\sigma+\partial_\beta A_\alpha\partial_\sigma A_\rho).\nonumber
\end{align}
Entonces
\begin{align}
  \frac{\partial}{\partial(\partial_\mu A_\nu)}F^{\rho\sigma}F_{\rho\sigma}=&g^{\rho\alpha}g^{\sigma\beta}(\delta_{\alpha\mu}\delta_{\beta\nu}\partial_\rho
  A_\sigma+\partial_\alpha A_\beta\delta_{\rho\mu} \delta_{\sigma\nu}-\delta_{\alpha\mu}\delta_{\beta\nu}\partial_\sigma A_\rho-\partial_\alpha A_\beta\delta_{\sigma\mu}\delta_{\rho\nu}\nonumber\\
&-\delta_{\beta\mu}\delta_{\alpha\nu}\partial_\rho A_\sigma-\partial_\beta A_\alpha\delta_{\rho\mu}\delta_{\sigma\nu}+\delta_{\beta\mu}\delta_{\alpha\nu}\partial_\sigma A_\rho+\partial_\beta
A_\alpha\delta_{\sigma\mu}\delta_{\rho\nu}).\nonumber\\
=&  g^{\rho\mu}g^{\sigma\nu}\partial_\rho A_\sigma+g^{\mu\alpha}g^{\nu\beta}\partial_\alpha A_\beta-g^{\rho\mu}g^{\sigma\nu}\partial_\sigma A_\rho-g^{\nu\alpha}g^{\mu\beta}\partial_\alpha A_\beta\nonumber\\
 &-g^{\rho\nu}g^{\sigma\mu}\partial_\rho A_\sigma-g^{\mu\alpha}g^{\nu\beta}\partial_\beta A_\alpha+g^{\rho\nu}g^{\sigma\mu}\partial_\sigma A_\rho+g^{\nu\alpha}g^{\mu\beta}\partial_\beta A_\alpha\nonumber\\
=&  \partial^\mu A^\nu+\partial^\mu A^\nu-\partial^\nu A^\mu-\partial^\nu A^\mu-\partial^\nu A^\mu-\partial^\nu A^\mu+\partial^\mu A^\nu+\partial^\mu A^\nu\nonumber\\
=&4(\partial^\mu A^\nu-\partial^\nu A^\mu) \nonumber\\
\label{eq:dddmufmunu2}
\frac{\partial}{\partial(\partial_\mu A_\nu)}F^{\rho\sigma}F_{\rho\sigma}&=4F^{\mu\nu}
\end{align}

Usando la ec.~\eqref{eq:dddmufmunu2}, tenemos
\begin{align}
\label{eq:177qft}
  \partial_\mu\left[
    \frac{\partial\mathcal{L}}{\partial(\partial_\mu A_\nu)}  
  \right]-\frac{\partial\mathcal{L}}{\partial A_\nu}&=0\nonumber\\
  -\frac{1}{4}\partial_\mu\left[
    \frac{\partial}{\partial(\partial_\mu A_\nu)}(F^{\rho\sigma}F_{\rho\sigma})  
  \right]+J^\rho\frac{\partial A_\rho}{\partial A_\nu}&=0\nonumber\\
  -\partial_\mu F^{\mu\nu}+J^\rho\delta_{\rho\nu}&=0\nonumber\\
  \partial_\mu F^{\mu\nu}&=J^\nu.
\end{align}
Como era de esperarse una Acción invariante de Lorentz e invariante
gauge local, expresada en términos del Lagrangiano \eqref{eq:lagAmum},
da lugar a la Teoría Electromagnética. 
 
Tomando la derivada con respecto a $\nu$ en ambos lados tenemos
\begin{align}
   \partial_\nu\partial_\mu F^{\mu\nu}&=\partial_\nu J^\nu.
\end{align}
De la parte izquierda de ésta ecuación tenemos
\begin{align*}
  \partial_\nu\partial_\mu F^{\mu\nu}&=\tfrac{1}{2}\left(\partial_\nu\partial_\mu F^{\mu\nu}+\partial_\nu\partial_\mu F^{\mu\nu}  \right)\nonumber\\
&=\tfrac{1}{2}\left(\partial_\nu\partial_\mu F^{\mu\nu}+\partial_\mu\partial_\nu F^{\nu\mu}  \right)
&&\text{intercambiando índices mudos}\nonumber\\
&=\tfrac{1}{2}\left(\partial_\nu\partial_\mu F^{\mu\nu}+\partial_\nu\partial_\mu F^{\nu\mu}  \right)
&&\text{conmutando derivadas}\nonumber\\
&=\tfrac{1}{2}\left(\partial_\nu\partial_\mu F^{\mu\nu}-\partial_\nu\partial_\mu F^{\mu\nu}  \right)
&&\text{usando antisimetría de $F^{\mu\nu}$}\nonumber\\
&=0\,,
\end{align*}
Por consiguiente, la cuadricorriente $J^\mu$ es conservada:
\begin{equation}
  \label{eq:consvjmu}
  \partial_\mu J^\mu=0\,.
\end{equation}


Again, for $\nu=0$, we have
\begin{align}
  \label{nohomME21}
    \partial_\mu F^{\mu0}&=J^0\nonumber\\
    \partial_iF^{i0}&=J^0\nonumber\\
    \frac{\partial}{\partial x^{i}}F^{i0}&=J^0\nonumber\\
    \frac{\partial E^{i}}{\partial x^{i}}&=J^0\,,
\end{align}
and therefore
\begin{align}
   \boldsymbol{\nabla}\cdot\mathbf{E}&=\rho\,.
\end{align}
while for $\nu=k$ we have

\begin{align}
\partial_\mu F^{\mu k}&=J^k\nonumber\\
\partial_iF^{ik}+\partial_0F^{0k}&=J^k\nonumber\\
-\partial_iF^{ki}-\partial_0F^{k0}&=J^k\nonumber\\
  -\frac{\partial (\epsilon_{ikj}B^j)}{\partial x^{i}}-\frac{\partial E^k}{\partial t}&=J^k\nonumber\\
\epsilon_{ijk}\frac{\partial B^j}{\partial x^{i}}-\frac{\partial E^k}{\partial t}&=J^k\nonumber\\
(\boldsymbol{\nabla}\times \mathbf{B})^k-\frac{\partial E^k}{\partial t}&=J^k.\,.
\end{align}
and therefore
\begin{align}
  \boldsymbol{\nabla}\times \mathbf{B}-\frac{\partial\mathbf{E}}{\partial t}&=\mathbf{J}\,.
\end{align}
In this way the expression
\begin{align}
   \partial_\mu F^{\mu\nu}&=J^\nu& \text{where}\quad    F^{\mu\nu}=\partial^\mu A^\nu-\partial^\nu A^\mu\,,
\end{align}
is completely equivalent to the full set of Maxwell equations:
\begin{align}
  \label{eq:hom_m_eq}
  \boldsymbol{\nabla}\cdot\mathbf{B}&=0,&\boldsymbol{\nabla}\times \mathbf{E}+\frac{\partial\mathbf{B}}{\partial t}&=0\\
  \label{eq:inhom_m_eq}
  \boldsymbol{\nabla}\cdot\mathbf{E}&=\rho,&\boldsymbol{\nabla}\times \mathbf{B}-\frac{\partial\mathbf{E}}{\partial t}&=\mathbf{J}\,.
\end{align}


%%%%%%%%%%%%%%%%%%%%%%%%%%%




De este modo las ecuaciones de Maxwell se pueden derivar del requerimiento de que la teor\'\i a, adem\'as de ser invariante de Lorentz, pueda expresarse en t\'erminos de potenciales de una forma que sea invariante gauge en esos potenciales. Si un cuadrivector potencial $A^\mu$ es postulado, y se impone que la teor\'\i a involucre este solamente, de una forma que sea insensible a a cambios de la forma \eqref{eq:aphicov}, se es conducido naturalmente a la idea de que los campos f\'\i sicos entran \'unicamente v\'\i a la cantidad $F^{\mu\nu}$, que es invariante bajo la ec. \eqref{eq:aphicov}. De aqu\'\i{} se puede conjeturar la ecuaci\'on de campo en base a la covarianza de Lorentz. 

Esto no corresponde ciertamente a una prueba de las ecuaciones de Maxwell. A pesar de eso, la idea que la din\'amica (en este caso la completa interconexi\'on entre los efectos el\'ectricos y magn\'eticos) pueda estar \'\i ntimamente relacionada a un requerimiento de invarianza local se ha convertido en algo muy fruct\'\i fero. 

En t\'erminos de transformaciones globales, se puede mostrar \cite{Aitchison:2003tq} que el cambio por una constante del potencial escalar ($\chi=at$, $A_0\to A_0'=A_0-a$, $\mathbf{A}\to \mathbf{A}'=\mathbf{A}$ en ec.\eqref{eq:phia_transf}), m\'as la conservaci\'on de energ\'\i a, da lugar a la conservaci\'on local de la carga. La conservaci\'on local en este contexto requiere que el cambio por una funci\'on del potencial escalar en en \eqref{eq:phia_transf} sea compensado por el correspondiente cambio en el vector potencial magn\'etico $A$. En general, cuando una cierta invarianza global es generalizada a una local, se requiere la existencia de un nuevo campo que compensa, interactuando de una manera espec\'\i fica. La teor\'\i as que dan lugar al Modelo Est\'andar y que describen las interacciones fuertes, d\'ebiles y electromagn\'eticas, son ejemplos de teor\'\i as din\'amicas derivadas desde un requerimiento de invarianza local.

Una de las principales razones
de porque la f\'\i sica de part\'\i culas se formula en t\'erminos de
lagrangianos, es que $\mathcal{L}$ debe ser escalar en cada espacio
relevante, e invariante bajo las transformaciones (hasta derivadas
totales), ya que la acci\'on es invariante. Haciendo el Lagrangiano
covariante de Lorentz por ejemplo, garantiza que todas las
predicciones son invariantes de Lorentz.



\subsection{Energ\'\i a del campo electromagn\'etico}
Necesitamos la expresi\'on para $F_{\mu\nu}$,
\begin{equation}
  \label{eq:16}
  F_{\mu\nu}=g_{\mu\rho}g_{\nu\eta}F^{\rho\eta}\Rightarrow
  \begin{cases}
    F_{0i}=F_{0\nu}=g_{00}g_{ij}F^{0j}=-F^{0i} &\text{para $\mu=0$}\\
    F_{ij}=F_{i\nu}=g_{ik}g_{jl}F^{kl}=F^{ij} &\text{para $\mu=i$}
  \end{cases}
\end{equation}
De la ec.~\eqref{eq:tmunu}, se tiene
\begin{align}
  T^\mu_\nu&=\frac{\partial\mathcal{L}}{\partial(\partial_\mu A_\lambda)}(\partial_\nu A_\lambda)
  -\delta^\mu_\nu\mathcal{L}\nonumber\\
  &=-F^{\mu\lambda}(\partial_\nu A_\lambda)-\delta^\mu_\nu\mathcal{L}
\end{align}
La energ\'\i a del campo, corresponde a la componente $T^0_0$:
\begin{align}
\label{eq:17}
  T^0_0&=-F^{0\lambda}(\partial_0A_\lambda)-\mathcal{L}\nonumber\\
  &=-F^{0\lambda}(\partial_0A_\lambda)+\frac{1}{4}F^{\mu\nu}F_{\mu\nu}+J^\mu A_\mu\nonumber
\end{align}
Usando las ecuaciones 
\eqref{eq:E_Fi0}, %noinstiki\eqref{eq:Efmunu},
\eqref{eq:BFij}, \eqref{eq:16}

\begin{align}
T^0_0 &=-F^{0\lambda}(\partial_0A_\lambda)+\frac{1}{4}F^{\mu\nu}F_{\mu\nu}+J^\mu A_\mu\nonumber\\
  &=-F^{0\mu}(\partial_0A_\mu)+\frac{1}{4}\overbrace{F^{\mu0}F_{\mu0}}^{\nu=0}+\frac{1}{4}\overbrace{F^{\mu i}F_{\mu i}}^{\nu=i}+J^\mu A_\mu\nonumber\\
  &=-F^{0\mu}\partial_\mu A_0-F^{\mu0}F_{\mu0}+\frac{1}{4}F^{\mu0}F_{\mu0}+\frac{1}{4}F^{\mu i}F_{\mu i}+J^\mu A_\mu\,.
\end{align}
Tenemos dos partes
\begin{align}
  -F^{\mu0}F_{\mu0}+\frac{1}{4}F^{\mu0}F_{\mu0}+\frac{1}{4}F^{\mu i}F_{\mu i}
  &=-F^{i0}F_{i0}+\frac{1}{4}F^{i0}F_{i0}+\frac{1}{4}\overbrace{F^{0i}F_{0i}}^{\mu=0}+\frac{1}{4}\overbrace{F^{ji}F_{ji}}^{\mu=j}\nonumber\\
  &=-F^{i0}F_{i0}+\frac{1}{4}F^{i0}F_{i0}+\frac{1}{4}{F^{i0}F_{i0}}+\frac{1}{4}{F^{ji}F_{ji}}\nonumber\\
  &=-\frac{1}{2}{F^{i0}F_{i0}}+\frac{1}{4}{F^{ji}F_{ji}}\,.
\end{align}
Adem\'as
\begin{align}
  -F^{0\mu}\partial_\mu A_0+J^\mu A_\mu=&-\partial_\mu(A_0 F^{0\mu})+A_0\partial_\mu F^{0\mu}+J^\mu A_\mu\nonumber\\
  =&-\partial_\mu(A_0 F^{0\mu})-A_0\partial_\mu F^{\mu0}+J^\mu A_\mu\nonumber\\
  =&-\partial_\mu(A_0 F^{0\mu})-A_0 J^0+J^\mu A_\mu\nonumber\\
  =&-\partial_i(A_0 F^{0i})-\mathbf{J}\cdot\mathbf{A}\,.
\end{align}
Entonces
\begin{align}
  T^0_0&=-\partial_i(A_0F^{0i})-\frac{1}{2}F^{i0}F_{i0}+\frac{1}{4}F^{ji}F_{ji}-\mathbf{J}\cdot\mathbf{A}\nonumber\\
 &=-\partial_i(A_0F^{0i})+\frac{1}{2}F^{i0}F^{i0}+\frac{1}{4}F^{ji}F^{ji}-\mathbf{J}\cdot\mathbf{A},\qquad\text{suma tambi\'en sobre $i,j$}\nonumber\\
  &=\frac{1}{2}E^{i}E^{i}+\frac{1}{4}\epsilon_{ijk}B^k\epsilon_{ijl}B^l+\partial_i(A_0E^{i})-\mathbf{J}\cdot\mathbf{A},\qquad\text{suma tambi\'en sobre $i,j$}\nonumber\\
  &=\frac{1}{2}\mathbf{E}^2+\frac{1}{2}\delta_{kl}B^k B^l+\boldsymbol{\nabla}\cdot(A^0\mathbf{E})-\mathbf{J}\cdot\mathbf{A}\nonumber\\
  &=\frac{1}{2}\mathbf{E}^2+\frac{1}{2}\mathbf{B}^2+\boldsymbol{\nabla}\cdot(A^0\mathbf{E})-\mathbf{J}\cdot\mathbf{A}
\end{align}

Entonces, en ausencia de corrientes
\begin{align}
  \mathcal{H}=&\frac{1}{2}\mathbf{E}^2+\frac{1}{2}\mathbf{B}^2+\boldsymbol{\nabla}\cdot(A^0\mathbf{E})\,.
\end{align}
Similarmente la densidad Lagrangiano puede escribirse como
\begin{align}
   \mathcal{L}=-\frac{1}{4}F^{\mu\nu}F_{\mu\nu}=\frac{1}{2}\left(\mathbf{E}^2-\mathbf{B}^2\right)
\end{align}
En vista a la ec.~\eqref{eq:17}, ya que la densidad Lagrangiana est\'a definida hasta una derivada total, como $\boldsymbol{\nabla}\cdot(A^0\mathbf{E})=\partial_\mu(A_0F^{\mu0})$, la densidad Hamiltoniana tambi\'en estar\'a definida hasta una derivada total. De hecho,
el Hamiltoniano es 
\begin{align}
  H&=\frac{1}{2}\int_Vd^3x\,(\mathbf{E}^2+\mathbf{B}^2)+ \int_Vd^3x\,\boldsymbol{\nabla}\cdot(A^0\mathbf{E})\nonumber\\
  \label{eq:18}
  &=\frac{1}{2}\int_Vd^3x\,(\mathbf{E}^2+\mathbf{B}^2),
\end{align}
y corresponde a la expresi\'on conocida para la energ\'\i a del campo electromagn\'etico. Hemos usado el hecho que en ausencia de corrientes todo lo que entra a un volum\'en debe salir y por consiguiente las integrales sobre el volumen de la divergencia de cualquier vector es cero.

Similarmente el momentum total del
campo, en ausencia de corrientes, corresponde al vector de Pointing:
%examen
\begin{align}
  T^0_i=&\frac{\partial\mathcal{L}}{\partial(\partial_0 A_\nu)}\partial_i A_\nu\nonumber\\
  =&-F^{0\nu}\partial_i A_\nu\nonumber\\
  =&-F^{0j}(\partial_i A_j-\partial_j A_i)-F^{0j}\partial_j A_i\nonumber\\
  =&-F^{0j}F_{ij}-F^{0j}\partial_j A_i\nonumber\\
  =&-F^{0j}F^{ij}-\partial_j (F^{0j}A_i)+(\partial_jF^{0j}) A_i\nonumber\\
  =&E^{j}\epsilon_{jik}B^k+\partial_j (E^jA_i)+(J^0) A_i\nonumber\\
  =&-(\mathbf{E}\times\mathbf{B})^i-\boldsymbol{\nabla}\cdot(A^i\mathbf{E})-\rho A^i\,
\end{align}
En ausencia de cargas y corrientes

\begin{align}
 P^i=-\int_Vd^3x\,T_{i}^0&=\int_Vd^3x\,(\mathbf{E}\times\mathbf{B})^i+\int_Vd^3x\,\boldsymbol{\nabla}\cdot(A^i\mathbf{E})\nonumber\\
 \label{eq:19}
 \mathbf{P}&=\int_Vd^3x\,(\mathbf{E}\times\mathbf{B})\,.
\end{align}

\subsection{Fijaci\'on del gauge}
\label{sec:fijacion-del-gauge}
%ver secci\'on 4.4 de Cottingan
Para obtener una soluci\'on definitiva a las ecuaciones del campo
electromagn\'etico, se debe remover la arbitrariedad asociada con la
libertad gauge de la ec.~\eqref{eq:aphicov}. De este modo los campos
quedan especificados un\'\i vocamente en todas partes. De hecho, de las
cuatro componentes del campo $A^\mu$, solo dos son independientes y
corresponden a los estados de polarizaci\'on de las ondas
electromagn\'eticas~\cite{Gross} (Cap\'\i tulo 2). A \'este proceso se le
denomina fijar el gauge, y consiste en imponer restricciones sobre los
campos que fijan la funci\'on  $\chi$ y remueven la libertad gauge. 

Nosotros usaremos el Gauge de Lorentz, definido por la condici\'on
\begin{equation}
  \label{eq:20}
  \partial_\mu A^\mu=0
\end{equation}
Si inicialmente $\partial_\mu A^\mu\neq0$, se realiza una transformaci\'on gauge tal
que $\partial_\mu A'^\mu=0$. De acuerdo a la ec.~\eqref{eq:aphicov}, esto da lugar
a la ecuaci\'on de onda inhomog\'enea
\begin{equation*}
\Box\chi=\partial_\mu A^\mu   
\end{equation*}
que puede solucionarse mediante las t\'ecnicas usuales. 

Es importante resaltar que la f\'\i sica queda inafectada por la escogencia
del gauge. El resultado final para cualquier observable f\'\i sico debe
ser independiente del gauge usado para calcularlo.

Las ecuaciones de Maxwell \eqref{eq:nohomME2} pueden escribirse como
\begin{align}
   \partial_\mu F^{\mu\nu}&=J^\nu\nonumber\\
  \partial_\mu(\partial^\mu A^\nu-\partial^\nu A^\mu)&=J^\nu\nonumber\\
  \partial_\mu\partial^\mu A^\nu-\partial_\mu\partial^\nu A^\mu&=J^\nu\nonumber\\
\label{eq:21}
  \partial_\mu\partial^\mu A^\nu-\partial^\nu(\partial_\mu A^\mu)&=J^\nu.
 \end{align}

Apliquemos ahora el gauge de Lorentz, ec.~(\ref{eq:20}) a las
ecuaciones inhomog\'eneas de Maxwell \eqref{eq:21}
\begin{equation}
  \label{eq:22} 
  \Box A^\nu=\partial_\mu\partial^\mu A^\nu=J^\nu.
\end{equation}
De este modo, cada componente del campo $A^\mu$ satisface la ecuaci\'on de
onda (\ref{eq:waveec}), o la ecuaci\'on de Klein-Gordon (\ref{eq:kg})
para masa cero. En ausencia de corrientes el campo $A^\mu$ puede ser
expandido en ondas planas con dos grados independientes de
polarizaci\'on~\cite{Gross}, de forma similar a como se hizo en la
secci\'on~\ref{sec:aplicacion-la-cuerda} para el campo $\phi$. Una vez
cuantizada la teor\'\i a, $A^\mu$ corresponde al fot\'on, y solo queda con dos
grados de libertad independientes que corresponden a los modos
transversales de la onda electromagn\'etica~\cite{Gross} (cap\'\i tulo 2).

La ec.~(\ref{eq:lagAmum}), with $J^\mu=0$, en el Gauge de Lorentz puede escribirse como
\begin{align}
  \label{eq:laglorgauge}
  \mathcal{L}=&-\frac{1}{4}F^{\mu\nu}F_{\mu\nu}\nonumber\\
  =&-\frac{1}{4}(\partial^\mu A^\nu-\partial^\nu A^\mu)(\partial_\mu A_\nu-\partial_\nu A_\mu)\nonumber\\
  =&-\frac{1}{4}(\partial^\mu A^\nu\partial_\mu A_\nu-\partial^\mu A^\nu\partial_\nu A_\mu-\partial^\nu A^\mu\partial_\mu A_\nu+\partial^\nu A^\mu\partial_\nu A_\mu)\nonumber\\
  =&-\frac{1}{4}[\partial^\mu A^\nu\partial_\mu A_\nu-\partial^\mu(A^\nu\partial_\nu A_\mu)+A^\nu\partial_\nu(\partial^\mu A_\mu)-\partial^\nu(A^\mu\partial_\mu A_\nu)+A^\mu\partial_\mu(\partial^\nu A_\nu)+\underbrace{\partial^\nu A^\mu\partial_\nu A_\mu}_{\mu\leftrightarrow\nu}]\nonumber\\
  =&-\frac{1}{4}[2\partial^\mu A^\nu\partial_\mu A_\nu-\partial^\mu(2A^\nu\partial_\nu A_\mu)]\nonumber\\
  =&-\frac{1}{2}\partial^\mu A^\nu\partial_\mu A_\nu
\end{align}
Incluyendo el t\'ermino con corrientes, y usando el hecho de que un signo global no afecta las ecuaciones de movimiento, tenemos
%to_en:Including the term with currents, and using the fact that a global sign will not affect the motion equations, we have
\begin{align}
  \label{eq:laglorgauugefin}
  \mathcal{L}=&\frac{1}{2}\partial^\mu A^\nu\partial_\mu A_\nu+J_\mu A^\mu
\end{align}

\section{Ecuaciones de Proca}
\label{sec:ecuacion-de-proca}
Consideraremos ahora el efecto de adicionar un t\'ermino de masa a la teor\'\i a de
Maxwell. Los campos vectoriales masivos juegan un papel importante en
f\'\i sica. Campos como $W^\mu$, $Z^\mu$ que median las interacciones d\'ebiles
son ejemplos de campos de este tipo. Las implicaciones de una masa
finita para el fot\'on pueden inferirse de un conjunto de postulados que
hacen de las ecuaciones de Proca la \'unica generalizaci\'on posible de
las ecuaciones de Maxwell \cite{Goldhaber:1971mr}. 

Teniendo en cuenta s\'olo el t\'ermino de masa en la ec.~(\ref{eq:lagAmum})
\begin{equation}
  \label{eq:23}
  \mathcal{L}=-\frac{1}{4}F^{\mu\nu}F_{\mu\nu}+\frac{1}{2}m^2A^\mu A_\mu-J^\mu A_\mu.
\end{equation}
Usando las ecuaciones de Euler-Lagrange, tenemos
\begin{align}
-\frac{1}{4}\partial_\mu
  \left[
\frac{\partial}{\partial(\partial_\mu A_\nu)}F^{\rho\eta}F_{\rho\eta}
  \right]-
\frac{\partial}{\partial A_\nu}
\left(
\frac{1}{2}m^2A^\rho A_\rho-J^\rho A_\rho
\right)&=0\nonumber\\
\label{eq:24}
\partial_\mu F^{\mu\nu}+m^2A^\nu&=J^\nu.
\end{align}
Tomando la cuadridivergencia a ambos lados de la ecuaci\'on y usando la
ec.~(\ref{eq:21}), tenemos
\begin{align}
 &\partial_\nu\partial_\mu\partial^\mu A^\nu-\partial_\nu\partial^\nu\partial_\mu A^\mu+m^2\partial_\nu A^\nu=\partial_\nu J^\nu\nonumber\\
 &\partial_\nu\partial_\mu\partial^\mu A^\nu-\partial_\mu\partial^\mu\partial_\nu A^\nu+m^2\partial_\nu A^\nu=\partial_\nu J^\nu\nonumber\\
\label{eq:25}
 &m^2\partial_\nu A^\nu=\partial_\nu J^\nu
\end{align}
De este modo, en ausencia de corrientes, la ecuaciones de Proca dan
lugar a la condici\'on de Lorentz. De otro lado, si asumimos que la
corriente se conserva, la condici\'on de Lorentz tambi\'en aparece. Por
consiguiente, si la masa de campo vectorial es diferente de cero, la
condici\'on de Lorentz, ec.~(\ref{eq:20}), emerge como una restricci\'on
adicional que debe ser siempre tomada en cuenta. De este modo la
libertad gauge de las ecuaciones de Maxwell se pierde completamente en
la ecuaciones de Proca, que sin perdida de generalidad se pueden
reescribir, usando $\partial_\mu A^\mu=0$ y las ecs.~(\ref{eq:21}),~(\ref{eq:24}),  como:
\begin{equation}
  \label{eq:26}
   (\Box+m^2)A^\nu=J^\nu
\end{equation}
En ausencia de corrientes, cada una de las componentes del campo
vectorial satisface la ecuaci\'on de Klein-Gordon~(\ref{eq:kg}). Por
consiguiente $m$ corresponde a la masa del campo vectorial
$A^\mu$. 

Aplicando la condici\'on de Lorentz a la ec.~(\ref{eq:23}), obtenemos el
Lagrangiano de la Ecuaci\'on de Proca (\ref{eq:26})
\begin{equation}
  \label{eq:27}
  \mathcal{L}=\frac{1}{2}\partial_\mu A^\nu\partial_\mu A^\nu-\frac{1}{2} m^2A^\nu A_\nu+J^\nu A_\nu,
\end{equation}
donde hemos reabsorbido un signo global que no afecta las ecuaciones
de movimiento. El primer t\'ermino que incluye s\'olo derivadas de los
campos es llamado \emph{t\'ermino cin\'etico} y dependen s\'olo del esp\'\i n de
las part\'\i culas. El t\'ermino cuadr\'atico en
los campos corresponde al \emph{t\'ermino de masa}, y el \'ultimo
corresponde a la interacci\'on del campo con una corriente. Cuando un
Lagrangiano contiene s\'olo t\'erminos cin\'eticos y de masa diremos que el
campo que da lugar al Lagrangiano es libre de interacciones, o
simplemente que es un \emph{campo libre}. Las otras partes del
Lagrangiano ser\'an llamadas \emph{Lagrangiano de Interacci\'on}. De este
modo podemos reescribir el Lagrangiano \eqref{eq:27} como
\begin{equation*}
\mathcal{L}=\mathcal{L}_{\text{free}}+\mathcal{L}_{\text{int}},  
\end{equation*}
donde,
\begin{align}
\mathcal{L}_{\text{free}}&=\frac{1}{2}\partial_\mu A^\nu\partial_\mu A^\nu-\frac{1}{2} m^2A^\nu A_\nu\nonumber\\
\label{eq:28}
\mathcal{L}_{\text{int}}&=J^\nu A_\nu.
\end{align}

Debido a que la teor\'\i a masiva ya no es invariante gauge, la condici\'on
de Lorentz aparece autom\'aticamente como la \'unica restricci\'on apropiada
sobre el campo vectorial.

Una vez se toma en cuenta la condici\'on de Lorentz el campo masivo
libre puede expandirse en ondas planas con tres grados de libertad
independientes de polarizaci\'on. Dos de estos corresponden a los dos
estados transversos que aparecen en las ondas electromagn\'eticas
($A^1$, $A^2$), y el tercero ($A^3$) corresponde a un estado
longitudinal en la direcci\'on del momento de la part\'\i cula \cite{Gross}.

Aunque hemos hecho el an\'alisis de la ecuaci\'on de Proca permitiendo un
t\'ermino de masa para el fot\'on, las implicaciones experimentales de una
teor\'\i a de este tipo dan lugar a restricciones muy fuertes sobre la
masa del fot\'on\cite{Goldhaber:1971mr}. El l\'\i mite actual sobre la masa
del fot\'on es $m\lt 6\times10^{-17}\,$eV ($1.1\times10^{-52}\,$Kg)
\cite{Yao:2006px}. Debido al principio gauge local, desde el punto
te\'orico se espera que la masa del fot\'on sea exactamente cero. En
general, los campos vectoriales puede ser generados a partir de otras
cargas no electromagn\'eticas y pueden ser masivos. El reto durante
varias d\'ecadas fue entender como las masa de los campos vectoriales de
la interacci\'on d\'ebil podr\'\i a hacerse compatible con el principio gauge
local.  






\section{Lorentz transformation of the fields}

Note again, that a term like
\begin{align}
\label{eq:nolor}
  \phi^*(x)a^\mu\partial_\mu\phi(x)\,,
\end{align}
does not left the Action invariant. To have a proper formulation of the quantum mechanics through the general equation
\begin{align}
  i\frac{\partial}{\partial t}\psi=\hat{H} \psi\,,  
\end{align}
with some, to be determined, relativistic Hamiltonian operator $\widehat{H}$, we should be able to build a Lagrangian with temporal derivatives of order one. Therefore, the Lorentz invariant requires all the derivatives of order one.  

Consider spinor fields, which transforms as
\begin{align}
\label{eq:184qft}
  \psi_a(x)\to\psi'_a(x)=S_{ab}(\Lambda)\psi_b(\Lambda^{-1}x)\,, 
\end{align}
where $S(\Lambda)$ is some spinorial representation of the Lorentz Group. We will check in next section if a Action with a term like
\begin{align}
  \psi^*_a(x)a^\mu_{ab}\partial_\mu\psi_b(x)
\end{align}
could be invariant under Lorentz transformations, for some internal representation of the Lorentz Group.

In summary we have the following Lorentz's transformation properties for the fields
\begin{align}
   \phi(x)\to \phi'(x')=&\phi(x) && \text{Scalar field,}\nonumber\\
   A^\mu(x)\to {A'}^\mu=&{\Lambda^\mu}_\nu A^\nu(\Lambda^{-1}x)&&\text{Vector field,}\nonumber\\
   \psi(x)\to\psi'(x)=&S(\Lambda)\psi(\Lambda^{-1}x)&&\text{Spinor field.}
\end{align}

\section{Dirac's Action}
\label{sec:dirac-equation}
The Scrodinger equation can be written as
\begin{align}
    i\frac{\partial}{\partial t}\psi=\hat{H}_{S} \psi\,,  
\end{align}
where
\begin{align}
  \hat{H}_{S}=
\end{align}


In order to have a well defined probabilty in relativistic quantum mechanics it is necessary that Lagrangian be linear in the time derivative, in order to obtain the general Sccödinger equation:
\begin{align}
  i\frac{\partial}{\partial t}\psi=\hat{H} \psi\,,  
\end{align}
like the Scrödinger Lagrangian. However, this automatically imply that the Lagrangian will be also linear in the spacial derivatives. A pure scalar field cannot involve a Lorentz invariant term of only first derivatives (see eq.~\eqref{eq:nolor}). Therefore the proposed field must have some internal structure associated with some representation of the Lorentz Group. Therefore we build the Lagrangian for a field of several components
\begin{align}
  \psi=  \begin{pmatrix}
\psi_1\\
\psi_2\\
\vdots\\
\psi_n    
  \end{pmatrix}
\end{align}

\subsection{Lorentz transformation}

If the field is to describe the electron. it must have spin and in this way it must transform under some spin representation of the Lorentz Group
\begin{align}
  \psi(x)\to \psi'(x)=S(\Lambda)\psi\left(\Lambda^{-1}x\right)\,.
\end{align}
One possible invariant could be the term $\psi^\dagger(x)\psi(x)$. However, under a Lorentz transformation we should have $\psi^\dagger S^\dagger S\psi$. As we cannot assume that $S(\Lambda)$ is unitary, the solution is to define the \emph{adjoint} spinor
\begin{align}
  \overline{\psi}=\psi^\dagger b\,.
\end{align}
which transforms as
\begin{align}
  \overline{\psi}(x)\to  \overline{\psi}'(x)&=
{\psi'}^\dagger(x)b=
\psi^\dagger\left(\Lambda^{-1}x\right)S^\dagger(\Lambda)b\,,
\end{align}
and,

\begin{align}
  \overline{\psi}(x)\psi(x)\to  \overline{\psi}'(x)\psi'(x)&=
\psi^\dagger\left(\Lambda^{-1}x\right)S^\dagger(\Lambda)b S(\Lambda)\psi\left(\Lambda^{-1}x\right)
\end{align}
The condition that must be fulfilled for Lorentz invariance of the Action is 
\begin{align}
  \label{eq:ltrinscal}
  S^\dagger(\Lambda)bS(\Lambda)=&b\,,
\end{align}
and therefore, 
\begin{align}
  \overline{\psi}(x)\psi(x)\to  \overline{\psi}'(x)\psi'(x)&=
\overline{\psi}\left(\Lambda^{-1}x\right)\psi\left(\Lambda^{-1}x\right)\,,
\end{align}
and:
\begin{align}
  \overline{\psi}(x)\to  \overline{\psi}'(x)&=
\psi^\dagger\left(\Lambda^{-1}x\right)b S^{-1}(\Lambda)\nonumber\\
&=\overline{\psi}\left(\Lambda^{-1}x\right)S^{-1}(\Lambda)\,.
\end{align}


A Action with a Lagrangian term linear in the derivatives, could be Lorentz invariant if, taking into account:
 \begin{align}
   \overline{\psi}(x)\gamma^\mu\partial_\mu\psi(x)\to  \overline{\psi'}(x)\gamma^\mu\partial_\mu\psi'(x)&=
 \overline{\psi}_a\left(\Lambda^{-1}x\right)S^{-1}_{ab}(\Lambda)\gamma^\mu_{bc}{\left(\Lambda^{-1}\right)^\rho}_\mu\partial_\rho S_{cd}(\Lambda)\psi_d\left(\Lambda^{-1}x\right)\nonumber\\
   &=
\overline{\psi} \psi\left(\Lambda^{-1}x\right){\left(\Lambda^{-1}\right)^\rho}_\mu \left(S^{-1}(\Lambda)\gamma^\mu S(\Lambda)\right)\partial_\rho\psi\left(\Lambda^{-1}x\right)\nonumber\\
&=\overline{\psi}(x)\gamma^\mu\partial_\mu\psi(x)\,,
 \end{align}
if the following condition is satisfied:
\begin{align}
\label{eq:ltrincond}
  S^{-1}(\Lambda)\gamma^\mu S(\Lambda)={\Lambda^\mu}_\sigma\gamma^\sigma\,.
\end{align}




the most general Lagrangian for this field is
\begin{align}
   \mathcal{L}&=i \overline{\psi} \gamma^\mu\partial_\mu\psi-m\overline{\psi} \psi\,,
\end{align}
Where the coefficients have been already fixed by convenience. Since the Action is real, it is convenient to rewrite this as
\begin{align}
   \mathcal{L}&=i \overline{\psi} \gamma^\mu\partial_\mu\psi-m\overline{\psi} \psi\nonumber\\
&=-\frac{1}{2}\partial_\mu\left(i \overline{\psi} \gamma^\mu\psi\right)+i \overline{\psi} \gamma^\mu\partial_\mu\psi-m\overline{\psi} \psi\nonumber\\
  &=-\frac{i}{2}(\partial_\mu \overline{\psi}) \gamma^\mu\psi-\frac{i}{2} \overline{\psi} \gamma^\mu\partial_\mu\psi+i \overline{\psi} \gamma^\mu\partial_\mu\psi-m\overline{\psi} \psi\nonumber\\
  &=\frac{i}{2} \overline{\psi} \gamma^\mu\partial_\mu\psi-\frac{i}{2}(\partial_\mu \overline{\psi}) \gamma^\mu\psi-m\overline{\psi} \psi\,.
\end{align}
 
Para que este nuevo Lagrangiano sea real se requiere que,
\begin{align}
  \label{eq:185qft}
  b^\dagger&=b\nonumber\\
  b^2&=I\nonumber\\
  b \gamma_\mu^\dagger b&=\gamma_\mu
\end{align}
ya que
\begin{align*}
  \mathcal{L}^\dagger&=\left(\frac{i}{2}\psi^\dagger \gamma_\mu^\dagger b \partial_\mu\psi-\frac{i}{2}\partial_\mu\psi^\dagger \gamma_\mu^\dagger b\psi\right)-m\psi^\dagger  b \psi\\
  &=\left(\frac{i}{2}\psi^\dagger b^2 \gamma_\mu^\dagger b \partial_\mu\psi-\frac{i}{2}\partial_\mu\psi^\dagger b^2 \gamma_\mu^\dagger b\psi\right)-m\psi^\dagger b \psi\\
  &=\left(\frac{i}{2}\bar{\psi} b \gamma_\mu^\dagger b \partial_\mu\psi-\frac{i}{2}\partial_\mu\bar{\psi}b \gamma_\mu^\dagger b\psi\right)-m\bar{\psi} \psi\\
  &=\left(\frac{i}{2}\bar{\psi} \gamma_\mu \partial_\mu\psi-\frac{i}{2}\partial_\mu\bar{\psi}\gamma_\mu \psi\right)-m\bar{\psi} \psi
\end{align*}

\subsection{Corriente conservada y Lagrangiano de Dirac}
\label{sec:corriente-conservada}
De la ec.~\eqref{eq:197qft}
\begin{align}
  J^0&=\left[\frac{\partial\mathcal{L}}{\partial\left(\partial_0\psi\right)}\right]\delta\psi+\delta\overline{\psi}\left[\frac{\partial\mathcal{L}}{\partial\left(\partial_0\overline{\psi}\right)}\right]\nonumber\\
  &=i\overline{\psi} \gamma^0 \delta\psi
\end{align}
El Lagrangiano es invariante bajo transformaciones de fase globales, $U(1)$
\begin{equation}
  \psi\to\psi'=e^{-i\alpha}\psi\approx\psi-i\alpha\psi,
\end{equation}
de modo que
\begin{equation}
  \delta\psi=-i\alpha\psi.
\end{equation}
Por consiguiente
\begin{equation}
  J^0=\alpha\overline{\psi} \gamma^0 \psi 
\end{equation}
Para que $J^0$ pueda interpretarse como una densidad de probabilidad, se debe cumplir
\begin{equation}
  \label{eq:bgamma0}
  b \gamma^0=I
\end{equation}


La  densidad de corriente es
\begin{align}
  J^0&\propto \psi^\dagger\psi\,.
\end{align}
Que podemos interpretar como una densidad de probabilidad.

De la ec.~\eqref{eq:bgamma0}, ya que la inversa de es única:
\begin{align}
  b=\gamma^0\,.
\end{align}
 
$\overline{\psi}$ se define como la \emph{adjunta} de $\psi$:
 \begin{align}
   \overline{\psi}=\psi^\dagger\gamma^0\,.
 \end{align}

It is convenient at this point to summarize the properties for $\gamma^0$:
\begin{align}
  \label{eq:cft77}
  {\gamma^0}^\dagger=&\gamma^0 & \left(\gamma^0\right)^2=&1 & \gamma^0{\gamma^\mu}^\dagger\gamma^0=&\gamma^\mu\nonumber\\
 &&   S^\dagger(\Lambda)\gamma^0S(\Lambda)=&\gamma^0\,. &&
\end{align}



En general
\begin{align}
   J^\mu&\propto\left[\frac{\partial\mathcal{L}}{\partial\left(\partial_\mu\psi\right)}\right]\delta\psi+\delta\bar{\psi}\left[\frac{\partial\mathcal{L}}{\partial\left(\partial_\mu\bar{\psi}\right)}\right]\nonumber\\
   &\propto i\bar{\psi}\gamma^\mu(-i\alpha\psi)\nonumber\\
   &\propto i\bar{\psi}\gamma^\mu(-i\alpha\psi)\nonumber\\
   &=\bar{\psi}\gamma^\mu\psi
\end{align}
y
\begin{equation}
     J^\mu=\psi^\dagger b \gamma^\mu\psi\,.
\end{equation}

\subsection{Tensor momento-energía}
\label{sec:tens-momento-energi}
\begin{align}
  T^0_0&=\frac{\partial\mathcal{L}}{\partial\left(\partial_0\psi\right)}\partial_0\psi+\partial_0\bar{\psi}\frac{\partial\mathcal{L}}{\partial\left(\partial_0\bar{\psi}\right)}-\mathcal{L}\nonumber\\
  &=i\bar{\psi}\gamma^0\partial_0\psi-\mathcal{L}\nonumber\\
  &=-i\bar{\psi}\gamma^i\partial_i\psi+m\bar{\psi} \psi,\nonumber\\
  &=\bar{\psi}(\boldsymbol{\gamma}\cdot\mathbf{p}+m)\psi,\nonumber\\
  &=\psi^\dagger \gamma^0(\boldsymbol{\gamma}\cdot\mathbf{p}+m)\psi,\nonumber\\
  \label{eq:118qft}
  &=\psi^\dagger\hat{H} \psi,
\end{align}
donde
\begin{equation}
  \label{eq:denshal}
  \hat{H}= \gamma^0(\boldsymbol{\gamma}\cdot\mathbf{p}+m)
\end{equation}
la ecuación de Scröndinger de validez general es entonces:
\begin{equation}
  i\frac{\partial}{\partial t}\psi=\hat{H} \psi
\end{equation}
y, como en mecánica clásica usual
\begin{equation}
  \label{eq:99qft}
  \langle\hat{H}\rangle=\int \psi^\dagger\hat{H} \psi\,d^3x.
\end{equation}


Además
\begin{align}
    T^0_i&=\frac{\partial\mathcal{L}}{\partial\left(\partial_0\psi\right)}\partial_i\psi+\partial_i\bar{\psi}\frac{\partial\mathcal{L}}{\partial\left(\partial_0\bar{\psi}\right)}\nonumber\\
    &=i\bar{\psi}\gamma^0 \partial_i\psi\nonumber\\
    &=-\psi^\dagger(-i\partial_i)\psi
\end{align}
de modo que
\begin{equation}
  \langle\hat{\mathbf{p}}\rangle=\int\psi^\dagger\hat{\mathbf{p}}\psi\,d^3 x
\end{equation}
\subsection{Ecuaciones de Euler-Lagrange}
\label{sec:ecuaciones-de-euler}
Queremos que el Lagrangiano de lugar a la ecuación de Scröndinger de validez general
\begin{equation}
  \label{eq:grlsch}
  i\frac{\partial}{\partial t}\psi=\hat{H} \psi
\end{equation}
con el Hamiltoniano dado en la ec.~(\ref{eq:99qft}), que corresponde a un Lagrangiano de sólo derivadas de primer orden y covariante, en lugar del Hamiltoniano para el caso no relativista. 

De hecho, aplicando las ecuaciones de Euler-Lagrange para el campo $\bar{\psi}$ al Lagrangiano en ec.~(\ref{eq:100qft}) ,tenemos
\begin{align}
  \partial_\mu\left[\frac{\partial\mathcal{L}}{\partial\left(\partial_\mu\bar{\psi}\right)}\right]-\frac{\partial\mathcal{L}}{\partial\bar{\psi}}&=0\nonumber\\
  \frac{\partial\mathcal{L}}{\partial\bar{\psi}}&=0\nonumber\\
  \label{eq:114qftm}
  i\gamma^\mu\partial_\mu\psi-m\psi&=0.
\end{align}
Expandiendo
\begin{align*}
  i\gamma^0\partial_0\psi+i\gamma^i\partial_i\psi-m\psi&=0\\
  i\gamma^0\partial_0\psi-\boldsymbol{\gamma}\cdot(-i\boldsymbol{\nabla})\psi-m\psi&=0,\\
  i\gamma^0\partial_0\psi&=(\boldsymbol{\gamma}\cdot\hat{\mathbf{p}}+m)\psi,
\end{align*}
de donde
\begin{equation}
    i{\gamma^0}^2\frac{\partial}{\partial t}\psi=\gamma^0(\boldsymbol{\gamma}\cdot\mathbf{p}+m)\psi.
\end{equation}
 tenemos que
\begin{align}
  \label{eq:gamma02}
  \left(\gamma^0\right)^2=1.
\end{align}
De la ec.~(\ref{eq:denshal})
\begin{equation}
  \label{eq:186qft}
  \hat{H}= \gamma^0(\boldsymbol{\gamma}\cdot\mathbf{p}+m),
\end{equation}
A este punto, sólo nos queda por determinar los parámetros $\gamma^\mu$. 

La ec.~(\ref{eq:grlsch}) puede escribirse como
\begin{equation}
  \left(i\frac{\partial}{\partial t}-\hat{H}\right)\psi=0.
\end{equation}
El campo $\psi$ también debe satisfacer la ecuación de Klein-Gordon. Podemos derivar dicha ecuación aplicando el operador
\begin{equation*}
  \left(-i\frac{\partial}{\partial t}-\hat{H}\right)
\end{equation*}
De modo que, teniendo en cuenta que $\partial\hat H/\partial t=0$,
\begin{align}
  \label{eq:105qft}
 \left(-i\frac{\partial}{\partial t}-\hat{H}\right)\left(i\frac{\partial}{\partial t}-\hat{H}\right)\psi&=0\nonumber\\
 \left(-i\frac{\partial}{\partial t}-\hat{H}\right)\left(i\frac{\partial\psi}{\partial t}-\hat{H}\psi\right)&=0\nonumber\\
 \frac{\partial^2\psi}{\partial t^2}+i\left(\frac{\partial\hat{H}}{\partial t}\right)\psi
 +i\hat{H}\frac{\partial\psi}{\partial t}-i\hat{H}\frac{\partial\psi}{\partial t}+\hat{H}^2\psi&=0\nonumber\\
 \left(\frac{\partial^2}{\partial t^2}+\hat{H}^2\right)\psi&=0.
\end{align}
% 
De la ec.~(\ref{eq:186qft}), y usando la condición en ec.~(\ref{eq:gamma02}), tenemos
\begin{align}
\label{eq:106qft}
\hat{H}^2&=(\gamma_0\boldsymbol{\gamma}\cdot\mathbf{p}+\gamma_0\,m)(\gamma_0\boldsymbol{\gamma}\cdot\mathbf{p}+\gamma_0\,m)\nonumber\\
&=(\gamma_0\boldsymbol{\gamma}\cdot\mathbf{p})(\gamma_0\boldsymbol{\gamma}\cdot\mathbf{p})+m\gamma_0\boldsymbol{\gamma}\cdot\mathbf{p}\gamma_0+m\gamma_0^2\boldsymbol{\gamma}\cdot\mathbf{p}+m^2
\end{align}
Sea
\begin{align}
  \beta&=\gamma^0\nonumber\\
  \alpha^i&=\beta\gamma^i\nonumber\\
  \gamma^i&=\beta\alpha^i
\end{align}
\begin{align}
  \hat{H}^2&=(\boldsymbol{\alpha}\cdot\mathbf{p})(\boldsymbol{\alpha}\cdot\mathbf{p})
  +m\boldsymbol{\alpha}\cdot\mathbf{p}\beta+m\beta\boldsymbol{\alpha}\cdot\mathbf{p}+m^2\nonumber\\
  &=(\boldsymbol{\alpha}\cdot\mathbf{p})(\boldsymbol{\alpha}\cdot\mathbf{p})
  +m(\boldsymbol{\alpha}\beta+\beta\boldsymbol{\alpha})\cdot\mathbf{p}+m^2
\end{align}
Sea $A$ una matriz y $\theta$ en un escalar. Entonces tenemos la identidad
\begin{align}
  \label{eq:206qft}
  (\mathbf{A}\cdot\boldsymbol{\theta})^2=\sum_i {A^i}^2 {\theta^i}^2+\sum_{i\lt j}\left\{A^i,A^j  \right\}\theta^i \theta^j 
\end{align}
\begin{itemize}
\item \textbf{Demostración}
  \begin{align}
    \left[\left(\mathbf{A}\cdot\boldsymbol{\theta}\right)\right]_{\alpha\beta}
    =&\sum_{i j}\sum_\gamma A^i_{\alpha\gamma}\theta^iA^j_{\gamma\beta}\theta^j\nonumber\\    
    =&\sum_{i j}\theta^i\theta^j\sum_\gamma A^i_{\alpha\gamma}A^j_{\gamma\beta}\nonumber\\    
    =&\sum_\gamma \sum_{i j}\theta^i\theta^jA^i_{\alpha\gamma}A^j_{\gamma\beta}\nonumber\\    
    =&\sum_\gamma \left(\sum_{i}{\theta^i}^2A^i_{\alpha\gamma}A^i_{\gamma\beta}+\sum_{i<j}\theta^i\theta^jA^i_{\alpha\gamma}A^j_{\gamma\beta}+\sum_{i>j}\theta^i\theta^jA^i_{\alpha\gamma}A^j_{\gamma\beta}\right)\nonumber\\    
    =&\sum_\gamma \left(\sum_{i}{\theta^i}^2A^i_{\alpha\gamma}A^i_{\gamma\beta}+\sum_{i<j}\theta^i\theta^jA^i_{\alpha\gamma}A^j_{\gamma\beta}+\sum_{j>i}\theta^j\theta^iA^j_{\alpha\gamma}A^i_{\gamma\beta}\right)\nonumber\\    
    =&\sum_\gamma \left[\sum_{i}{\theta^i}^2A^i_{\alpha\gamma}A^i_{\gamma\beta}+\sum_{i<j}\theta^i\theta^j\left(A^i_{\alpha\gamma}A^j_{\gamma\beta}+A^j_{\alpha\gamma}A^i_{\gamma\beta}\right)\right]\nonumber\\    
    =&\left[\sum_{i}{\theta^i}^2\left(A^iA^i\right)_{\alpha\beta}+\sum_{i<j}\theta^i\theta^j\left\{ A^i,A^j\right\}_{\alpha\beta}\right]\nonumber\\    
    =&\left[\sum_{i}{\theta^i}^2{A^i}^2+\sum_{i<j}\theta^i\theta^j\left\{ A^i,A^j\right\}\right]_{\alpha\beta}\,.
  \end{align}

\end{itemize}
Entonces
\begin{align}
  \hat{H}^2=&\alpha_i^2p_i^2+\sum_{i\lt j}\left\{\alpha_i,\alpha_j\right\}p_i p_j+m(\alpha_i \beta+\beta\alpha_i)p_i+m^2
\end{align}
(suma sobre índices repetidos). Si
\begin{align}
  \label{eq:107qft}
  \alpha_i^2&=1\nonumber\\
  \left\{\alpha_i,\alpha_j\right\}&=0\qquad i\neq j\nonumber\\
  \alpha_i \beta+\beta\alpha_i&=0
\end{align}
\begin{equation}
  \hat{H}^2=-\boldsymbol{\nabla}^2+m^2
\end{equation}
y reemplazando en la ec.~\eqref{eq:105qft} llegamos a la ecuación de Klein-Gordon para $\psi$
\begin{align}
   \left(\frac{\partial^2}{\partial t^2}-\boldsymbol{\nabla}^2+m^2\right)\psi&=0\nonumber\\
   \left(\Box+m^2\right)\psi&=0
\end{align}
En términos de las matrices $\gamma^\mu$ las condiciones en ec.~\eqref{eq:107qft} son
\begin{align}
  \label{eq:108qft}
  \left({\gamma^0}\right)^2&=1\nonumber\\
  \left({\alpha^i}\right)^2=1\to\gamma^0\gamma^i \gamma^0\gamma^i=-\left({\gamma^i}\right)^2=1\to\left({\gamma^i}\right)^2&=-1\nonumber\\
  \gamma^i \gamma^0+\gamma^0\gamma^i=\left\{\gamma^i,\gamma^0\right\}&=0
\end{align}
De modo que
\begin{align}
  \label{eq:198qft}
\left\{\alpha^i,\alpha^j\right\}=\gamma^0\gamma^i \gamma^0\gamma^j+\gamma^0\gamma^j \gamma^0\gamma^i&=0\qquad i\neq j\nonumber\\
-\gamma^0\gamma^0\gamma^i \gamma^j-\gamma^0\gamma^0\gamma^j \gamma^i&=0\qquad i\neq j\nonumber\\
\gamma^i \gamma^j+\gamma^j \gamma^i&=0\qquad i\neq j\nonumber\\
\left\{\gamma^i,\gamma^j\right\}&=0\qquad i\neq j
\end{align}
Las ecuaciones \eqref{eq:108qft}\eqref{eq:198qft} pueden escribirse como
\begin{equation}
  \label{eq:109qft}
  \left\{\gamma^\mu,\gamma^\nu\right\}\equiv\gamma^\mu\gamma^\nu+\gamma^\nu\gamma^\mu=2g^{\mu\nu}\mathbf{1}
\end{equation}
donde
\begin{align}
  \gamma^\mu=(\gamma^0,\gamma^i)
\end{align}
Además, de la ec.~\eqref{eq:cft77}
\begin{equation}
  \label{eq:112qft}
   \gamma^0{\gamma^\mu}^\dagger \gamma^0=\gamma^\mu.
\end{equation}
Cualquier conjunto de matrices que satisfagan el álgebra en ec.~\eqref{eq:109qft} y la condición en ec.~\eqref{eq:112qft}, se conocen como matrices de Dirac. A $\psi$ se le llama espinor de Dirac.

En términos de la matrices $\gamma^\mu$, el Lagrangiano de Dirac y la ecuación de Dirac, son respectivamente de las ecs.~(\ref{eq:100qft}) y (\ref{eq:114qft})
\begin{equation}
  \label{eq:115qft}
  \mathcal{L}=\bar{\psi}\left(i\gamma^\mu\partial_\mu-m\right)\psi,
\end{equation}
\begin{equation}
  \label{eq:116qft}
  i\gamma^\mu\partial_\mu\psi-m\psi=0,
\end{equation}
donde
\begin{equation}
  \bar{\psi}=\psi^\dagger\gamma^0.
\end{equation}





\subsection{Propiedades de las matrices de Dirac}
\label{sec:propiedades-de-las}
De la ec.~(\ref{eq:112qft})
\begin{equation}
  {\gamma^\mu}^\dagger=\gamma^0\gamma^\mu\gamma^0\Rightarrow  
  \begin{cases}
    {\gamma^0}^\dagger=\gamma^0&\mu=0\\
    {\gamma^i}^\dagger=-{\gamma^0}^2\gamma^i=-\gamma^i&\mu=i
  \end{cases}.
\end{equation}
Definiendo
\begin{equation}
\label{eq:117qft}
  \gamma_5=i\gamma_0\gamma_1\gamma_2\gamma_3,
\end{equation}
entonces,
\begin{align}
  \gamma_5^2=&-\gamma_0\gamma_1\gamma_2\gamma_3\gamma_0\gamma_1\gamma_2\gamma_3\nonumber\\
  \gamma_5^2=&+\gamma_0^2\gamma_1\gamma_2\gamma_3\gamma_1\gamma_2\gamma_3\nonumber\\
  \gamma_5^2=&+\gamma_1\gamma_2\gamma_3\gamma_1\gamma_2\gamma_3\nonumber\\
  \gamma_5^2=&-\gamma_2\gamma_3\gamma_2\gamma_3\nonumber\\
  \gamma_5^2=&\gamma_2\gamma_2\gamma_3\gamma_3\nonumber\\
  \gamma_5^2=&\mathbf{1}\,.
\end{align}

\begin{equation}
  \gamma_5^2=\mathbf{1},
\end{equation}
Teniendo en cuenta que $\gamma_\mu^2\propto\mathbf{1}$ y conmuta con las demás matrices, tenemos por ejemplo
\begin{align}
  \gamma_5\gamma_3=&i\gamma_0\gamma_1\gamma_2\gamma_3^2=\gamma_3^2i\gamma_0\gamma_1\gamma_2=-\gamma_3i\gamma_0\gamma_1\gamma_2\gamma_3=-\gamma_3\gamma_5\nonumber\\
  \gamma_5\gamma_2=&-i\gamma_0\gamma_1\gamma_2^2\gamma_3=-\gamma_2^2i\gamma_0\gamma_1\gamma_3=-\gamma_2i\gamma_0\gamma_1\gamma_2\gamma_3=-\gamma_2\gamma_5\nonumber\\
  \gamma_5\gamma_1=&i\gamma_0\gamma_1^2\gamma_2\gamma_3=\gamma_1^2i\gamma_0\gamma_2\gamma_3=-\gamma_1i\gamma_0\gamma_1\gamma_2\gamma_3=-\gamma_1\gamma_5\nonumber\\
  \gamma_5\gamma_0=&i\gamma_0\gamma_1\gamma_2\gamma_3\gamma_0=-\gamma_0^2i\gamma_1\gamma_2\gamma_3=-\gamma_0\gamma_5\,.
\end{align}
De modo que
\begin{equation}
  \label{eq:218qft}
  \left\{\gamma_\mu,\gamma_5\right\}=0. 
\end{equation}
Expandiendo el anticonmutador tenemos
\begin{align}
  \gamma_\mu\gamma_5=-\gamma_5\gamma_\mu\nonumber\\
  \gamma_5\gamma_\mu\gamma_5=-\gamma_\mu\nonumber\\
\operatorname{Tr}\left(\gamma_5\gamma_\mu\gamma_5\right)=-\operatorname{Tr}\gamma_\mu\nonumber\\
\operatorname{Tr}\left(\gamma_5\gamma_5\gamma_\mu\right)=-\operatorname{Tr}\gamma_\mu\nonumber\\
\operatorname{Tr}\gamma_\mu=-\operatorname{Tr}\gamma_\mu,
\end{align}
y por consiguiente
\begin{equation}
  \operatorname{Tr}\gamma_\mu=0.
\end{equation}


De otro lado, si
\begin{equation}
  \tilde{\gamma_\mu}\equiv U\gamma_\mu U^\dagger,
\end{equation}
para alguna matriz unitaria $U$, entonces $\tilde{\gamma_\mu}$ corresponde a otra representación de álgebra de Dirac en ec.~(\ref{eq:109qft}), ya que
\begin{align}
  \left\{\tilde\gamma^\mu,\tilde\gamma^\nu\right\}&=\left\{U\gamma^\mu U^\dagger,U\gamma^\nu U^\dagger\right\}\nonumber\\
  &=U\left\{\gamma^\mu,\gamma^\nu\right\}U^\dagger\nonumber\\
  &=2g^{\mu\nu}UU^\dagger\nonumber\\
  &=2g^{\mu\nu}\mathbf{1}.
\end{align}
Claramente, la condición en ec.~(\ref{eq:112qft}) se mantiene para la nueva representación. Como $\gamma_0$ es hermítica, siempre es posible escoger una representación tal que $\tilde{\gamma_0}\equiv U\gamma_0U^\dagger$ sea diagonal. Como $\gamma_0^2=1$, sus entradas en la diagonal deben ser $\pm1$, y como $\operatorname{Tr}\tilde\gamma_0=0$, debe existir igual número de $+1$ que de $-1$. Por lo tanto la dimensión de $\gamma_0$ (y de $\gamma_\mu$) debe ser par: $2,4,\ldots$. Para un fermion sin masa
\begin{align}
  \mathcal{L}=i\psi^\dagger\gamma^0\gamma^0\partial_0\psi+i\psi^\dagger\gamma^0\gamma^i\partial_i\psi=i\psi^\dagger\partial_0\psi+i\psi^\dagger\alpha^i\partial_i\psi\,,
\end{align}
solo se requieren tres matrices $2\times 2$ que satisfacen
\begin{align}
  \left\{\alpha^i,\alpha^j\right\} =2\delta^{ij}\,,
\end{align}
y por lo tanto pueden identificarse con las tres matrices de Pauli. 
Como en general tenemos 4 matrices independientes, su dimensión mínima debe ser 4.

Como $\tilde\gamma^i=\gamma^0\gamma^i\gamma^0={\gamma^i}^\dagger=-\gamma^i$, podemos definir la \emph{representación de paridad}
\begin{align}
\label{eq:parityrep}
\tilde\gamma^0=&\gamma^0,\qquad\tilde\gamma^i=-\gamma^i\,,&\text{para}\qquad U=&\gamma^0   
\end{align}




\begin{inprogress}
  \subsection{Dirac representation}
The set of $4\times 4$ matrices
\begin{align}
  S^{\mu\nu}=\frac{i}{4}\left[\gamma^\mu,\gamma^\nu\right]\,,
\end{align}
also satisfy the commutations relations \eqref{eq:lrtalg}, and constitute a new matrix representation of the Lorentz Group. The subgroup of rotation group has the generators $S^{ij}$. We define the spin matrices:
\begin{align}
  \Sigma^i=\epsilon^{ijk}S_{jk}\,,
\end{align}
taking into account that
\begin{align}
  \gamma_0\gamma_i\gamma_5=-\epsilon_{0ijk}S^{jk}\,,
\end{align}
Taking th convention $\epsilon_{0ijk}=\epsilon_{ijk}$, we have
\begin{align}
  \Sigma_{i}=-\gamma_0\gamma_i\gamma_5\,.
\end{align}
With this form, it is easy to show that
\begin{align}
  \left\{\Sigma_i,\Sigma_j\right\}=2\delta_{ij}\,, 
\end{align}
so that 
\begin{align}
  \left[\frac{\Sigma^i}{2},\frac{\Sigma^j}{2} \right]=i\,\epsilon_{ijk}\frac{\Sigma^k}{2}
\end{align}
\end{inprogress}


\subsection{Lorentz Group}
We must build a representation of the Lorentz Group in the Dirac space of $n$ dimensions. First, let us consider a simpler group, corresponding to the rotation group in tree dimensions. The generators are the angular momentum operators $J^i$, which satisfy the commutation relations
\begin{align}
\label{eq:rotgr}
  \left[J^i,J^j\right]=i\epsilon^{ijk}J^k\,.
\end{align}
The Pauli matrices are set of matrices satisfying this commutation relations:
\begin{equation}
  \label{eq:paulialg}
  \left[\frac{\tau^i}{2},\frac{\tau^j}{2} \right]=i\,\epsilon_{ijk}\frac{\tau^k}{2}
\end{equation}
donde $\tau^i$ 
\begin{equation}
  \label{eq:paulimatr}
  \tau_1=
  \begin{pmatrix}
    0&1\\
    1&0
  \end{pmatrix} \qquad
 \tau_2=
  \begin{pmatrix}
    0&-i\\
    i&0
  \end{pmatrix}\qquad 
 \tau_3=
  \begin{pmatrix}
    1&0\\
    0&-1
  \end{pmatrix}
 \end{equation}
dividas por dos, corresponden a los generadores del Grupo. Las constantes de estructura del Grupo corresponden a $\epsilon_{ijk}$. Como los generadores no conmutan, $SU(2)$ es un Grupo de Lie no Abeliano. Definiendo los generadores de $SU(2)$ como
\begin{equation}
  T^i=\frac{\tau_i}{2},
\end{equation}
un elemento del Grupo puede escribirse como
\begin{equation}
  \label{eq:63qft}
  U=e^{iT^i \theta_i }\approx1+iT^i\theta_i=1+i\frac{\tau^i}{2}\theta_i\,.
\end{equation}
Como antes, $\theta_i$ es el parámetro de la transformación. 

Las matrices de Pauli y por consiguiente $T_i$ satisfacen 
\begin{align}
  \tau_i^\dagger&=\tau_i\nonumber\\
  \operatorname{Tr}  \left(
    \tau_i
  \right)&=0
\end{align}
Además
\begin{align}
  \label{eq:64qft}
  \det
  \left(
    \tau_i
  \right)&=-1\nonumber\\
  \left\{ 
    \tau_i,\tau_j
  \right\}&=2\delta_{ij}\cdot I\Rightarrow\tau_i^2=I\nonumber \\
\operatorname{Tr} \left(\tau^i\tau^j\right)&=2\delta^{ij}\nonumber\\
\tau_i\tau_j&=i\epsilon_{ijk}\tau_k+\delta_{ij}
\end{align}
In \cite{Peskin}:
\begin{quote}
  It is generally true that one can find matrix representations of a continuous group by finding matrix representations of the generators of the group (which must satisfy the proper commutation relations), then exponentiating these infinitesimal representations. 

For our present problem, we need to know the commutation relations of the generators of the group of Lorentz transformations. For the rotation group, one can work the commutation relations by writing the generators as differential operators; from the expression
\begin{align}
  \mathbf{J}=\mathbf{x}\times \mathbf{p}=\mathbf{x}\times (-i\boldsymbol{\nabla})\,,
\end{align}
the angular momentum commutation relations \eqref{eq:rotgr} follow straightforwardly. 
\end{quote}
The last equation can be written as (summation of repeated indices)
\begin{align}
  J^k=\left[\mathbf{x}\times (-i\boldsymbol{\nabla})\right]^k=&
-i\epsilon_{ijk}x^i\partial_j=i\epsilon_{ijk}x^i\partial^j
\end{align}
\begin{align}
  J^{l m}\equiv\epsilon_{lmk}J^k=&i\epsilon_{lmk}\epsilon_{ijk}x^i\partial^j\nonumber\\
=&i(\delta_{li}\delta_{mj}-\delta_{lj}\delta_{mi})x^i\partial^j\nonumber\\
=&i(x^l\partial^m-x^m\partial^l)\,.
\end{align}
Involving three generators. The generalization to four-dimensions give to arise three further generators $J^{0i}$:
\begin{align}
  J^{\mu\nu}=i(x^\mu\partial^\nu-x^\nu\partial^\mu)\,.
\end{align}
The six generators $J^{\mu\nu}$ satisfy the algebra
\begin{align}
\label{eq:lrtalg}
  \left[J^{\mu\nu},J^{\rho\sigma}\right]=&
i(g^{\nu\rho}J^{\mu\sigma}-g^{\mu\rho}J^{\nu\sigma}-g^{\nu\sigma}J^{\mu\rho}+g^{\mu\sigma}J^{\nu\rho})\,.
\end{align}
From \cite{Peskin}:
\begin{quote}
  Any matrices that are to represent this algebra must obey these same commutation rules. 
\end{quote}

The exponentiation of the generators give to arise to group elements
\begin{align}
  \Lambda=\exp\left(-i\omega_{\mu\nu}\frac{J^{\mu\nu}}{2}\right)
\end{align}


To find a representation of the usual boosts and rotations, 
consider a boost
\begin{equation}
  \left\{x^\mu\right\}=\begin{pmatrix}
    t\\
    x\\
    y\\
    z
  \end{pmatrix}\to
  \begin{pmatrix}
    t'\\
    x'\\
    y'\\
    z'
  \end{pmatrix}=
  \begin{pmatrix}
    \frac{t+vx}{\sqrt{1-v^2}}\\
    \frac{x+vt}{\sqrt{1-v^2}}\\
    y\\
    z
  \end{pmatrix}=
  \begin{pmatrix}
    \cosh\xi&\sinh\xi&0&0\\
    \sinh\xi&\cosh\xi&0&0\\
    0     &  0  &1&0\\
    0     &  0  &0&1
  \end{pmatrix}
  \begin{pmatrix}
    t\\
    x\\
    y\\
    z
  \end{pmatrix}=\left\{{\Lambda^\mu}_{\nu}\right\}\left\{x^\nu\right\},
\end{equation}
Since
\begin{align}
  \cosh\xi=&\sum_{n=0}^{\infty}\frac{\xi^{2n}}{2n!}\approx 1+\mathcal{O}(\xi^2)\nonumber\\
  \sinh\xi=&\sum_{n=0}^{\infty}\frac{\xi^{2n+1}}{(2n+1)!}\approx \xi+\mathcal{O}(\xi^2)\,,
\end{align}
one infinitesimal boost along $x$ is
\begin{align}
  \left\{{\Lambda^\mu}_{\nu}\right\}_{x-\text{boost }}\approx
  \begin{pmatrix}
    1&\xi&0&0\\
    \xi&1&0&0\\
    0&0&1&0\\
    0&0&0&1
  \end{pmatrix}.
\end{align}
Similarly a rotation by an infinitesimal angle $\theta=\theta_3$ along $xy$--plane (or about the $z$--axis)
\begin{align}
  \left\{{\Lambda^\mu}_{\nu}\right\}_{xy-\text{rotation }}\approx
  \begin{pmatrix}
    1&0&0&0\\
    0&1&\theta&0\\
    0&-\theta&1&0\\
    0&0&0&1
  \end{pmatrix}.
\end{align}
In general we define the six independent Lorentz--Group parameters:
\begin{align}
  \omega_{0i}=-\omega_{i0}\equiv&\xi_i \nonumber\\
  \omega_{12}=-\omega_{21}\equiv&\theta_3 &   \omega_{32}=-\omega_{23}\equiv&-\theta_2 &   \omega_{13}=-\omega_{31}\equiv&\theta_1\,.
\end{align}
The $4\times 4$ matrices
\begin{align}
  \left(J^{\mu\nu}\right)_{\alpha\beta}=i\left({\delta^\mu}_\alpha{\delta^\nu}_\beta-{\delta^\nu}_\beta{\delta^\mu}_\alpha\right)\,,
\end{align}
where $\mu$ and $\nu$ label which of the six matrices we want, while $\alpha$ and $\beta$ label components of the matrices. These matrices satisfy the commutations relations \eqref{eq:lrtalg}, and generate the three boosts and three rotations of the ordinary Lorentz 4-vectors:
\begin{align}
  {\Lambda^\alpha}_\beta\approx{\delta^\alpha}_\beta-\frac{i}{2}\omega_{\mu\nu}{\left(J^{\mu\nu}\right)^\alpha}_\beta
\end{align}
%ver programa mathematica
\begin{align}
\label{eq:lorentzrep}
  \Lambda=1+\xi_ib^i+\frac{1}{2}\theta_i\epsilon_{i j k}r^{jk}\,,
\end{align}
\begin{align}
 b^i=&-i J^{i0} & r^{jk}=-i J^{j k}\,.
\end{align}

\subsection{Lorentz invariance of the Dirac Action}
We need to satisfy the following conditions
\begin{align}
  S^{-1}(\Lambda) \gamma^\mu S(\Lambda)=&{\Lambda^\mu}_\nu\gamma^\nu\nonumber\\
  S^\dagger(\Lambda) \gamma^0 S(\Lambda)=&\gamma^0\qquad\text{or}\quad S^\dagger(\Lambda) \gamma^0= \gamma^0S^{-1}(\Lambda)\, .
\end{align}

In order to find a representation of the Lorentz Group in terms of the Dirac matrices we propose
  \begin{align}
    \label{eq:diraclorentzrep}
  S(\Lambda)=1+\xi_iB^i+\frac{1}{2}\theta_i\epsilon_{i j k}R^{jk}\,.
\end{align}
Instead of show the Lorentz invariance of the Dirac Action, we use the conditions derived from the invariance, to find a representation in terms of the Dirac matrices for $B^i$ and $R^{jk}$. As a consistency check, the resulting representation would satisfy the Lorentz algebra. In this way, by using eq.~\eqref{eq:lorentzrep} and \eqref{eq:diraclorentzrep}, we obtain from 
\begin{align}
  S^{-1}(\Lambda)\gamma^\mu S(\Lambda)={\Lambda^\mu}_\nu\gamma^\nu\,,
\end{align}
that
\begin{align}
  B^i=\frac{1}{2}\gamma^0\gamma^i\nonumber\\
  R^{jk}=&\frac{1}{2}\gamma^j\gamma^k\,,
\end{align}
which can be written in covariant form if we define
\begin{align}
  \mathcal{S}^{\mu\nu}=\frac{i}{4}\left[\gamma^\mu,\gamma^\nu\right]\,.
\end{align}
In fact, the six set of non-zero independently generators are
\begin{align}
  \mathcal{S}^{0i}=&\frac{i}{4}\left(\gamma^0\gamma^i-\gamma^i\gamma^0\right)=\frac{i}{2}\gamma^0\gamma^i= i B^i\nonumber\\
  \mathcal{S}^{i j}=&\frac{i}{4}\left(\gamma^i\gamma^j-\gamma^j\gamma^i\right)=\frac{i}{2}\gamma^i\gamma^j= i R^{i j}\,.
\end{align}
It is worth notices that in fact $\mathcal{S}^{\mu\nu}$ satisfy the Lorentz algebra, and therefore are the generators of the Lorentz group elements:
\begin{align}
  S(\Lambda)=&\exp\left(-i \omega_{\mu\nu}\frac{\mathcal{S}^{\mu\nu}}{2}\right)\nonumber\\
  \approx&1-\frac{i}{2} \omega_{\mu\nu}{\mathcal{S}^{\mu\nu}}\,.
\end{align}
Another consistency check is
\begin{align}
  S^\dagger(\Lambda)\gamma^0S(\Lambda)=&\gamma^0\,,
\end{align}
or equivalently
\begin{align}
S^\dagger(\Lambda)\gamma^0=&\gamma^0S^{-1}(\Lambda)\nonumber\\
\left(1+\frac{i}{2} \omega_{\mu\nu}{\mathcal{S}^{\mu\nu}}^\dagger \right)\gamma^0=&\gamma^0\left(1+\frac{i}{2} \omega_{\mu\nu}{\mathcal{S}^{\mu\nu}}\right)\nonumber\\
{\mathcal{S}^{\mu\nu}}^\dagger \gamma^0=&\gamma^0{\mathcal{S}^{\mu\nu}}\,.
\end{align}
Taking into account that
\begin{align}
  {\gamma^\mu}^\dagger{\gamma^\nu}^\dagger\gamma^0=\left(\gamma^0\right)^2{\gamma^\mu}^\dagger\left(\gamma^0\right)^2{\gamma^\nu}^\dagger\gamma^0=\gamma^0\gamma^\mu\gamma^\nu\,,
\end{align}
we have
\begin{align}
  {\mathcal{S}^{\mu\nu}}^\dagger \gamma^0=&-\frac{i}{4}\left[\gamma^\mu,\gamma^\nu\right]^\dagger\gamma^0\nonumber\\
=&-\frac{i}{4}\left[{\gamma^\nu}^\dagger,{\gamma^\mu}^\dagger\right]\gamma^0\nonumber\\
=&\frac{i}{4}\left[{\gamma^\mu}^\dagger,{\gamma^\nu}^\dagger\right]\gamma^0\nonumber\\
=&\frac{i}{4}\left[{\gamma^\mu},{\gamma^\nu}\right]\gamma^0\nonumber\\
=&\gamma^0\mathcal{S}^{\mu\nu}\nonumber\\
\end{align}


\subsection{Dirac's Lagrangian}
\label{sec:diracs-lagrangian}

Para una matriz de $n$ dimensiones existen $n^2$ matrices hermíticas (o anti--hermíticas) independientes. Si se sustrae la identidad quedan $n^2-1$ matrices hermíticas (o anti--hermíticas) independientes de traza nula. En el caso $n=2$ corresponden a las 3 matrices de Pauli. En el caso de la ecuación de Dirac se requieren 4 matrices independientes, por lo tanto deben ser matrices $4\times 4$. En efecto para $n=4$ existen 15 matrices independientes de traza nula dentro de las cuales podemos acomodar sin problemas las 4 $\gamma^\mu$. 

De \cite{Gross:1993}:
\begin{quote}
  All Dirac matrix elements will now be written in the form
  \begin{align}
    \overline{\psi}(x)\Gamma\psi(x)\,,
  \end{align}
where $\Gamma$ is a $4\times 4$ complex matrix. The most general such matrix can always be expanded in terms of 16 independent $4\times 4$ matrices multiplied by complex coefficients. In short the matrices $\Gamma$ can be regarded as a \emph{16--dimensional complex vector space} spanned by 16 matrices.

It is convenient to choose the 16 matrices, $\Gamma_i$, so that they have well defined transformation properties under the Lorentz Transformations. Since the $\gamma^\mu$'s have such properties, we are lead to choose the following 16 matrices for this basis:
\end{quote}


En la Tabla~\ref{tab:Gamma} se muestran las matrices de traza nula con sus propiedades de transformación bajo el Grupo de Lorentz. En la última se muestra el correspondiente escalar en el espacio de Dirac $\bar\psi\Gamma\psi$.
%instiki:
\begin{table} %noinstiki
  \centering %noinstiki
  \begin{tabular}{l|l|l|l} %noinstiki
Matriz $\Gamma$&Transformación&Número&Escalar en Dirac\\\hline{}
%instiki:
$\mathbf{1}$&Escalar (S)&1&$\bar\psi\psi$\\
%instiki:
$\gamma_5$&Pseudoescalar (P)&1&$\bar\psi\gamma_5\psi$\\
%instiki:
$\gamma_\mu$&Vector (V)&4&$\bar\psi\gamma_\mu\psi$\\
%instiki:
$\gamma_\mu\gamma_5$ &Vector axial (A)&4&$\bar\psi\gamma_\mu\gamma_5\psi$\\
%instiki:
$\sigma_{\mu\nu}=\frac{i}{2}\left[\gamma_\mu,\gamma_\nu\right]$&Tensor antisimétrico (T)&6&$\bar\psi\sigma_{\mu\nu}\psi$\\\hline{}
%instiki:
&&16&\\
  \end{tabular} %noinstiki
  \caption{Matrices $\Gamma_i$.} %noinstiki
\label{tab:Gamma} %noinstiki
\end{table} %noinstiki
%instiki:
Demostración
\begin{align}
J^\mu(x)\equiv  \bar\psi(x)\gamma^\mu\psi(x)\to&\bar\psi(\Lambda^{-1}x)S^{-1}(\Lambda)\gamma^\mu S(\Lambda)\psi(\Lambda^{-1}x) \nonumber\\
=&{\Lambda^\mu}_\nu\bar\psi(\Lambda^{-1}x) \gamma^\nu\psi(\Lambda^{-1}x) \nonumber\\
=&{\Lambda^\mu}_\nu J^\nu(\Lambda^{-1}x)\,.
\end{align}
In \cite{Gross:1993}: Problem 5.4: 
\begin{align}
  \overline{\psi}\gamma_5\psi\to\overline{\psi}S^{-1}(\Lambda)\gamma^5S(\Lambda)\psi =(\det\Lambda)\overline{\psi}\gamma_5\psi
\end{align}
The solution is in Appendix C. of Burgess book, by using
\begin{align}
  \gamma^5=\frac{i}{24}\epsilon_{\mu\nu\alpha\beta}\gamma^\mu\gamma^\nu\gamma^\alpha\gamma^\beta
\end{align}
and
\begin{align}
  \det \Lambda=\epsilon_{\mu\nu\alpha\beta}{\Lambda^\mu}_1{\Lambda^\nu}_2{\Lambda^\alpha}_3{\Lambda^\beta}_4\,.
\end{align}




\section{Problemas}
\label{sec:problemas2}
\renewcommand{\labelenumi}{\thechapter.\theenumi} %noinstiki

\begin{enumerate} %noinstiki
\item Muestre que
  \begin{equation*}
    {\Lambda^{\mu}}_{\nu}{\Lambda_\mu}^{\rho}={\Lambda_{\mu}}^{\nu}{\Lambda^{\mu}}_{\rho}=\delta^\rho_\nu
  \end{equation*}
Compruebe esta identidad para la transformaci\'on de Lorentz de la ec.~\eqref{eq:147}
\label{item:pch2.1} %noinstiki

\item Muestre que el Lagrangiano electromagn\'etico en ausencia de corrientes
  \begin{equation}
    \mathcal{L}=-\frac{1}{4}F^{\mu\nu}F_{\mu\nu}=\frac{1}{2}\left(\mathbf{E}^2-\mathbf{B}^2\right)
  \end{equation}
\label{item:pch2.2} %noinstiki

\item Calcule el rango de la interacci\'on d\'ebil mediada por la part\'\i cula $W^{-}$ de masa
  \begin{equation}
    m_W\approx80\;GeV
  \end{equation}
\label{item:pch2.3} %noinstiki
\end{enumerate} %noinstiki
\renewcommand{\labelenumi}{\theenumi} %noinstiki
%%% Local Variables: 
%%% mode: latex
%%% TeX-master: "fullnotes"
%%% End: 

